\documentclass{swfusyllabus}

\addbibresource{syllabus.bib}

\title{Linux应用}

\swfusetup{%
  enTitle={Linux Basics},%
  ID={50001198},% 课程编号
  Term=1,% 第3学期
  Points={2/32},% 学分/总学时
  Lecture={16},% 理论学时
  Lab={16},% 实验学时
  Cls={必修课},% 课程性质
  Type={专业基础},% 课程类别
  Pre={无},%先修课程
  Book={clj2006},%建议教材
  Books={clj2006,cooper2010advanced},% 参考教材和资料
  Sbj={电子信息工程、计算机科学与技术、数据科学与大数据技术},% 适用专业
  School={大数据与智能工程学院},%
  Apply={2022本科人才培养方案},%适用范围
  Exam={考试},% 考核方式
}


\begin{document}

\maketitle{}

\section{课程基本信息}

\basicinfo{}

\section{课程简介}

\subsection{课程性质}
% 该门课程在本专业的性质、地位、及涉及对本专业教学贡献度范围等……。

目前,国家正在以Linux系统为基础大力推进国产操作系统的研发与应用。《操作
系统原理》以及越来越多的其它课程都在采用Linux为平台开展教学工作。因此要
求学生具有较好的Linux应用基础能力。

\subsection{课程任务}

% 该课程主要通过课堂教学讲述什么理论内容、对什么原理进行分析、采用何种教
% 学方法、重点掌握什么、一般了解什么等……。通过上述……理解和掌握,使学生理
% 解……方法,为后续课程的学习打下良好的基础。

% 如果有课内实验:通过何种教学实验,验证什么原理与理论、进而掌握何种知识、
% 最后达到何种设计能力与创新水平等……。

% 撰写说明:课程性质和课程任务可参考原教学大纲内容。

本课程先简要介绍GNU/Linux的历史及发展前景,然后,采用理论与实践并重的教
学手段,面向软件开发,重点介绍Bash命令行的应用,软件开发环境的使用,以
及基本的Linux系统管理和网络管理常识。通过大量的动手实验,使学生了
解Linux平台上的软件开发常识,并具备一定的实际动手能力,为后续课程的学习
打下良好的基础。

\subsection{课程思政}
% 说明:以下列举了课程思政内容,以及举例了教学内容融合点的描述。教师根据
% 课程教学内容可部分选择、增删修改以上思政内容。不要求全部选择以上思政内
% 容,不要求是完全和以上思政内容一致。

本课程在课程教学中将思想政治教育内容与课程专业知识教育进行融合:

\begin{political}{lX}%{colspec={lX},hlines,vlines,row{1}=c}
  教学内容& 课程思政内容及融入点\\
  GNU/Linux的历史与未来&理想信念教育(建立自信,激发志向/立足专业,勇攀
  高峰:)通过对我国当前在操作系统研发领域的成果进行介绍,帮助学
  生建立民族自信心,激发学生树立中华民族复兴的志向。\\
  Bash命令行& 工匠精神:结合命令行工具的学习,将工匠精神逐步渗透到教学
  中,培养学生认真踏实
  的学习态度,充分发挥课程育人的功能。\\
  软件开发环境& 可持续发展:结合软件开发环境的使用、更新,让学生意识到
  软件开发的可持续性。\\
\end{political}

\section{课程目标}

% 说明:
% \begin{enumerate}
% \item 毕业要求 12 条分解的指标点(附件一 XX 专业毕业要求指标点分
%   解,由各系系主任提供)。课程目标需要支撑毕业要求中的分解指标点。
       
% \item 根据参考资料,建议课程目标的数量为 2--10 个,一般为 3-6 个。课程目
%   标对应的分解指标点数量为 1-5 个,一般为 3-4 个。1 个课程目标可以对
%   应 1 个及以上分解指标点;1 个以上课程目标可以只对应 1 个分解指标点。
      
% \item 课程目标文字描述,可采用适当的动词引导:了解/理解……概念/原理/特性
%   等;熟悉/掌握/拓宽/提高/获得/……理论/知识/方法/工具/技术/流程/管理等;
%   完成/具备/分析/解决/培养/锻炼/运用……训练/能力/设计开发等。
% \end{enumerate}
% 参考:
% \begin{itemize}
% \item 《附件一 XX 专业毕业要求指标点分解》中课程目标文字描述。
% \item 《XX 课程教学大纲》示范
% \end{itemize}

\subsection{课程目标}
% 以下文字仅做参考,教师实际撰写课程目标不限于上面文字约束和数量约束。

课程目标及能力要求具体如下:

\begin{enumerate}
\item 知识与技能目标:通过本课程的学习,使学生在已有的计算机知识的基础
  上,了解GNU/Linux的历史与发展方向,理解Shell解释器的基
  本原理、基本特性,掌握基本命令的使用。
  % 该部分主要撰写掌握何种的基本概念、基本原理、基本特性
  % 及设计方法,并具有应用相关原理和方法进行复杂工程问题中的综合和设计的
  % 能力等……;
\item 过程与方法目标:通过教师讲授、实验、课外上机实践等环节,学生
  在Linux应用的学习过程中,对Linux系统从整体上有清晰全面的系统了解,了
  解Linux平台上最为流行的软件开发工具,掌握Vim、Emacs等编辑器的基本使用
  方法,理解软件编译的概念,掌握gcc编译器的基本使用方法。了解软件包管理
  的概念,掌握Debian系统中常用的软件包管理命令,以及最基本的网络管理命
  令。
  % 该部分主要撰写掌握该课程的涉及到的结构、机构、程序、
  % 工艺、流程等进行原理分析,采用何种方法进行训练与求解,最后达到掌握何
  % 种知识、提高何种能力……;
\item 情感、态度与价值观发展目标:通过本课程的学习,掌握Linux系统
  应用的思维分析方法和基本工具,培养积极思考、严谨创新的科学态度和
  解决实际问题的能力,培养使用计算机操作系统知识和方法解决计算机科学领
  域相关实际问题的能力。
  % 该部分主要撰写能够使用相关仪器设备开展常用机构、结构、
  % 程序和流程等设计与分析的实验研究,对涉及到的方案进行创新设计……;
% \item[课程目标 4:] 该部分主要撰写具备查阅相关文献的能力……;
\end{enumerate}

\subsection{课程目标对毕业要求的支撑关系}

\begin{support}{ccXl}%{colspec={ccXl},vlines,hlines,row{1}={c,m}}
  % 课程目标 m:& 指标点 x.y& 即教学组织,通过讲授、演示、提问、设计、编
  % 程、实践操作、翻转课堂、项目驱动等方式,让学生理解、掌握、完
  % 成……知识/能力/训练等&
  % 作业、实验、考试、设计报告、专题研究等形式,根据课程自行设定依据\\
  %------------------------------------------------------------------------------------------------------------------------------------------
  {课程\\目标}& {毕业要求\\指标点}& 达成途径& 评价依据\\
  1& {指标点\\5.3}&%
  通过讲授、演示、提问、设计、编程、实践操作、翻转课堂、项目驱动等方式,
  让学生了解GNU/Linux的历史及发展方向、理解Shell解释器的基本原理、基本特性,掌握基本命令的使用。&%
  {作业、实验、\\考试}\\
  2& {指标点\\5.3}& %
  通过讲授、演示、提问、设计、编程、实践操作、翻转课堂、项目驱动等方式,
  让学生了解Linux平台上最为流行的软件开发工具,掌握Vim、Emacs等编辑器的
  基本使用方法,理解软件编译的概念,掌握gcc编译器的基本使用方法。&%
  {作业、实验、\\考试}\\
  3& {指标点\\5.3}&%
  通过讲授、演示、提问、设计、编程、实践操作、翻转课堂、项目驱动等方式,
  让学生了解软件包管理的概念,掌握Debian系统中常用的软件包管理命令,以
  及最基本的网络管理命令。&%
  {作业、实验、\\考试}\\
\end{support}

\section{教学内容、教学要求及学时分配}

\subsection{理论教学}
% 说明:课程思政内容与课程简介中描述要一致;课程思政内容不是每一章或者
% 每次课程都要融入。

\begin{lecture}{XXlcc}%{colspec={XXlcc},hlines,vlines,row{1}={c,m}}
  % {按章节或按教学顺序\\1.1\\1.2\\课程思政:} &%
  % 了解……理解……掌握……&%
  % {1. 课堂讲授\\ 2. 多媒体演示\\ 3. 任务驱动\\ 4. 翻转课堂\\5. 归纳总
  %   结\\ 6. 小组研讨/报告\\ 7. 答疑与互动\\ ……\\ 以上方
  %   法根据教学选择}&&\\
  % --------------------------------------------------------------------------------------------------------------------------------------------------------------------
  课程教学内容& 教学要求& 教学设计& {推荐\\学时}& {支撑课程\\目标}\\
  %%%%%%%%%%%%%%%%%%%%%%%%%%%%%%%%%%%%%%%%%%%%%%%%
  {%课程教学内容
    1. GNU/Linux和开源运动\\
    1.1 什么是开源\\
    1.2 什么是 GNU/Linux?\\
    1.3 开源软件能干什么?\\
    1.4 怎样学习 Linux?\\
    课程思政:理想信念教育}&%
  {%教学要求
    了解GNU/Linux的历史与发展方向\\通过对我国当前在操作系统研发领域的成
    果进行介绍,帮助学生建立民族自信心,激发学生树立中华民族复兴的志向。建立自信,激发志向/立足专业,勇攀高峰}&
  {%教学设计
    1. 课堂讲授\\
    2. 多媒体演示\\
    3. 答疑与互动}&%
  2&%推荐学时
  {指标点\\5.3}\\%支撑课程目标
  %%%%%%%%%%%%%%%%%%%%%%%%%%%%%%%%%%%%%%%%%%%%%%%%
  {%
    2. Bash命令行\\
    2.1 什么是解释器?\\
    2.2 什么是命令行?\\
    2.3 常用命令介绍\\
    课程思政:工匠精神}&%
  {理解Shell解释器的基本原理、基本特性,掌握基本命令的使用。\\结合命令行工具的学习,将工匠精神逐步渗透到教学中,培养学生认真踏实的学习态度,充分发挥课程育人的功能。}&%
  {%
    1. 课堂讲授\\
    2. 多媒体演示\\
    3. 任务驱动\\
    4. 答疑与互动\\
    5. 归纳总结}& 4&{指标点\\5.3} \\
  %%%%%%%%%%%%%%%%%%%%%%%%%%%%%%%%%%%%%%%%%%%%%%%%
  {%
    3. Bash编程\\
    3.1 Vim编辑器\\
    3.2 Emacs编辑器\\
    3.3 Bash语法\\
    3.4 程序示例}&%
  能读懂简单的Bash程序;能编写简单的Bash程序&%
  {%
    1. 课堂讲授\\
    2. 多媒体演示\\
    3. 任务驱动\\
    4. 答疑与互动\\
    5. 归纳总结}& 4&{指标点\\5.3} \\
  %%%%%%%%%%%%%%%%%%%%%%%%%%%%%%%%%%%%%%%%%%%%%%%%
  {%
    4. C开发环境\\
    4.1 编译器的工作原理\\
    4.2 gcc的使用\\
    4.3 gdb的使用}&%
  理解软件编译的概念;掌握gcc编译器的基本使用方法。&%
  {%
    1. 课堂讲授\\
    2. 多媒体演示\\
    3. 任务驱动\\
    4. 答疑与互动\\
    5. 归纳总结}& 4&{指标点\\5.3} \\
  %%%%%%%%%%%%%%%%%%%%%%%%%%%%%%%%%%%%%%%%%%%%%%%%
  {
    5. 系统管理与网络管理\\
    5.1 常用Debian软件包管理命令\\
    5.2 常用网络命令}&%
  了解软件包管理的概念,掌握Debian系统中常用的软件包管理命
  令,以及最基本的网络管理命令。&%
  {
    1. 课堂讲授\\
    2. 多媒体演示\\
    3. 任务驱动\\
    4. 答疑与互动\\
    5. 归纳总结}&2&{指标点\\5.3} \\
  %课程教学内容& 教学要求& 教学设计& 推荐学时& 支撑课程目标\\
\end{lecture}

\subsection{教学方法}
% 以下教学方法,请根据实际教学,说明教学方法,以及采用该教学方法达到什么目的。

\begin{enumerate}
\item 课堂讲授与讨论:本课程的理论部分,主要采取课堂讲授为主,将课程中
  所涉及到的背景、概念、思想、方法以深入浅出的语言介绍给学生,并鼓励学
  生参与互动讨论,鼓励学生提问。
\item 多媒体演示:以图片(框图、流程图、时序图、程序示例等)为主,一图
  胜千言,围绕图片、示例展开解说,事半而功倍。
\item 任务驱动:在教学过程中,结合教学内容,适时地给学生布置一些小任务,
  以提高学生的教学参与度,使学生在完成小任务的过程中,加强对教学内容的
  理解。
\item 答疑互动:在教学过程中,鼓励学生提问,同时适时地向学生提问,以提
  高学生的参与度,启发学生思考,加强学生对教学内容的理解。
\end{enumerate}

\subsection{实验教学}%(如果没有课内实验,本部分内容删除)
% 实验类型为:演示型、验证型、综合型、设计型、创新型。
% 实验要求为:必做、选做。

实验教学是本课程中的重要实践环节,目的是培养学生的动手能力,让学生尽快
熟悉Linux系统中的基本命令操作,以及编程工具的使用,同时加深学生
对Linux系统概念与原理的理解。本课程的实验教学为非独立设课,具体要求如
下。

\begin{lab}{lXcccc}%{colspec={lXcccc},hlines,vlines,row{1}={c,m}}
  实验项目名称& 教学要求& {实验\\学时}& {每组\\人数}& {实验\\类型}&{实验\\要求}\\
  {命令行的基本操作}&%
  {熟练使用基本命令}&%
  4&1&综合型&必做\\
  {编辑器的使用}&%
  {能够使用Vim、Emacs编写简单程序}&%
  4&1&综合型&必做\\
  {Bash编程}&%
  {能够读懂简单Bash程序;能够编写简单Bash程序}&%
  4&1&综合型&必做\\
  {系统管理}&%
  {能够使用Debian命令安装、卸载、更新软件包}&%
  2&1&综合型&必做\\
  {网络管理}&%
  {能够使用网络命令诊断并排除简单的网络故障}&%
  2&1&综合型&必做\\
\end{lab}

\section{课程的考核环节}

\subsection{成绩评定法}
% 说明:以下考核项目、考核项目百分比根据课程评价依据而定。

\begin{itemize}
\item \(期末总成绩=平时成绩(25\%)+实验成绩(25\%)+期末卷面成绩(50\%)\)
\item \(平时成绩=考勤成绩(10\%)+课堂表现(40\%)+作业成绩(50\%)\)
\item \(实验成绩 = 实验过程成绩(50\%)+ 实验报告成绩(50\%)\)
\item \(期末卷面成绩 = 期末考试卷面成绩\)
\end{itemize}

\section{参考教材和资料}
% [1] 《xxxxxxxxxx》(第 x 版)[M]. xx:xxxxx 出版社, xxx.
% [2] 《xxxxxxxxxx》(第 x 版)[M]. xx:xxxxx 出版社, xxx.
% [3] 《xxxxxxxxxx》(第 x 版)[M]. xx:xxxxx 出版社, xxx.

% \begin{refsection}
%   \nocite{tanenbaum2011computer,fall2011tcp,kurose2013computer,%
%     bautts2005linux,hunt2002tcp,hall2009beej}
%   \printbibliography[heading=none]{}
% \end{refsection}

\booklist{}

\vfill
\begin{flushright}
  执笔人签字:\includegraphics[width=25mm]{wangxiaolin}\qquad 审稿人签字:\makebox[2cm][c]{}\\
  主管教学院长签字:\makebox[2cm][c]{}
\end{flushright}

\end{document}
%%% Local Variables:
%%% mode: latex
%%% TeX-master: t
%%% End:
