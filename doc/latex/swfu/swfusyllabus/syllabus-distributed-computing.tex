\documentclass{swfusyllabus}

\addbibresource{syllabus.bib}

\title{分布式计算概论}

\swfusetup{%
  enTitle={Introduction To Distributed Computing},%
  ID={50000960},% 课程编号
  Term=4,% 第4学期
  Points={3/48},% 学分/总学时
  Lecture={24},% 理论学时
  Lab={24},% 实验学时
  Cls={选修修课},% 课程性质
  Type={专业特色},% 课程类别
  Pre={数据结构,操作系统原理,计算机网络},%先修课程
  Book={xsk2024},%建议教材
  Books={xsk2024,coulouris2013,tanenbaum2011computer,tanenbaum2015modern},% 参考教材和资料
  Sbj={电子信息工程、计算机科学与技术、数据科学与大数据技术},% 适用专业
  School={大数据与智能工程学院},%
  Apply={2022本科人才培养方案},%适用范围
  Exam={考试},% 考核方式
}

\begin{document}

\maketitle{}

\section{课程基本信息}

\basicinfo{}

\section{课程简介}

\subsection{课程性质}
% 该门课程在本专业的性质、地位、及涉及对本专业教学贡献度范围等……。

《分布式系统概论》是计算机科学与技术及相关专业的专业课程,是计算机科学
与技术学科基本理论和知识体系的重要组成部分,兼具理论性和实践性。通过本
课程的学习,使学生掌握分布式计算及分布式系统的基本理论、经典算法和关键
技术,充分了解当今分布式计算的最新研究成果和典型分布式系统架构,建立系
统、算法和应用的系统化知识体系。通过典型程序演示和案例驱动提高学生在分
布式计算的学术研究和工程研发水平。

\subsection{课程任务}

% 该课程主要通过课堂教学讲述什么理论内容、对什么原理进行分析、采用何种教
% 学方法、重点掌握什么、一般了解什么等……。通过上述……理解和掌握,使学生理
% 解……方法,为后续课程的学习打下良好的基础。

% 如果有课内实验:通过何种教学实验,验证什么原理与理论、进而掌握何种知识、
% 最后达到何种设计能力与创新水平等……。

% 撰写说明:课程性质和课程任务可参考原教学大纲内容。

通过本课程的学习,使学生掌握分布式计算及分布式系统的基本理论、经典算法
和关键技术,充分了解当今分布式计算的最新研究成果和典型分布式系统架构,
建立系统、算法和应用的系统化知识体系。通过典型程序演示和案例驱动提高学
生在分布式计算的学术研究和工程研发水平。

\subsection{课程思政}
% 说明:以下列举了课程思政内容,以及举例了教学内容融合点的描述。教师根据
% 课程教学内容可部分选择、增删修改以上思政内容。不要求全部选择以上思政内
% 容,不要求是完全和以上思政内容一致。

本课程在课程教学中将思想政治教育内容与课程专业知识教育进行融合:

\begin{political}{lX}%{colspec={lX},hlines,vlines,row{1}=c}
  教学内容& 课程思政内容及融入点\\
  分布式系统简介&理想信念教育(建立自信,激发志向/立足专业,勇攀高峰):
  通过对我国当前在网络研发领域的成果进行介绍,帮助学
  生建立民族自信心,激发学生树立中华民族复兴的志向。\\
  分布式系统模型& 工匠精神:结合网络协议设计思想的学习,将工匠精神逐步渗透到教
  学中,培养学生认真踏实的学习态度,充分发挥课程育人的功能。\\
  分布式系统的通信方法& 可持续发展:结合IP协议的升级问题,让学生意识到协议开
  发的可持续性。\\
\end{political}

\section{课程目标}

% 说明:
% \begin{enumerate}
% \item 毕业要求 12 条分解的指标点(附件一 XX 专业毕业要求指标点分
%   解,由各系系主任提供)。课程目标需要支撑毕业要求中的分解指标点。
       
% \item 根据参考资料,建议课程目标的数量为 2--10 个,一般为 3-6 个。课程目
%   标对应的分解指标点数量为 1-5 个,一般为 3-4 个。1 个课程目标可以对
%   应 1 个及以上分解指标点;1 个以上课程目标可以只对应 1 个分解指标点。
      
% \item 课程目标文字描述,可采用适当的动词引导:了解/理解……概念/原理/特性
%   等;熟悉/掌握/拓宽/提高/获得/……理论/知识/方法/工具/技术/流程/管理等;
%   完成/具备/分析/解决/培养/锻炼/运用……训练/能力/设计开发等。
% \end{enumerate}
% 参考:
% \begin{itemize}
% \item 《附件一 XX 专业毕业要求指标点分解》中课程目标文字描述。
% \item 《XX 课程教学大纲》示范
% \end{itemize}

\subsection{课程目标}
% 以下文字仅做参考,教师实际撰写课程目标不限于上面文字约束和数量约束。

课程目标及能力要求具体如下:
\begin{enumerate}
\item 知识与技能目标:通过本课程的学习,使学生掌握分布式计算及分布式系
  统的基本理论、经典算法和关键技术,充分了解当今分布式计算的最新研究成
  果和典型分布式系统架构,建立系统、算法和应用的系统化知识体系。通过典
  型程序演示和案例驱动提高学生在分布式计算的学术研究和工程研发水平。

\item 过程与方法目标:通过教师讲授、实验、课外上机实践等环节,学生在计
  算机分布式系统的学习过程中,系统掌握分布式系统的思维分析
  方法、基本概念、重要思想。在此基础上进行归纳和总结,逐步形成和掌握运
  用计算机网络知识完成分布式系统应用开发的学习观和方法论。
  
\item 情感、态度与价值观发展目标:通过本课程的学习,掌握分布式系统的思
  维分析方法和基本分析工具,培养积极思考、严谨创新的科学态度和解决实际
  问题的能力,培养使用计算机网络知识和方法解决计算机科学领域相关实际问
  题的能力。
\end{enumerate}

\subsection{课程目标对毕业要求的支撑关系}

\begin{support}{ccXl}%{colspec={ccXl},vlines,hlines,row{1}={c,m}}
  % 课程目标 m:& 指标点 x.y& 即教学组织,通过讲授、演示、提问、设计、编
  % 程、实践操作、翻转课堂、项目驱动等方式,让学生理解、掌握、完
  % 成……知识/能力/训练等&
  % 作业、实验、考试、设计报告、专题研究等形式,根据课程自行设定依据\\
  %------------------------------------------------------------------------------------------------------------------------------------------
  {课程\\目标}& {毕业要求\\指标点}& 达成途径& 评价依据\\
  %------------------------------------------------------------------------------------------------------------------------------------------
  1& {指标点\\11.2}&%
  通过讲授、演示、提问、设计、编程、实践操作、翻转课堂、项目驱动等方式,
  让学生理解分布式系统的概念与工作原理。&%
  {作业、实验、\\考试}\\
  %------------------------------------------------------------------------------------------------------------------------------------------
  2& {指标点\\12.2}& %
  通过讲授、演示、提问、设计、编程、实践操作、翻转课堂、项目驱动等方式,
  帮助学生在分布式系统设计思想和方法的学习过程中,系统掌握思维分析
  方法、基本概念、重要思想。在此基础上进行归纳和总结,逐步形成和掌握运
  用网络知识完成分布式系统应用开发的学习观和方法论。&%
  {作业、实验、\\考试}\\
  %------------------------------------------------------------------------------------------------------------------------------------------
  3& {指标点\\12.2}&%
  通过讲授、演示、提问、设计、编程、实践操作、翻转课堂、项目驱动等方式,
  培养学生积极思考、严谨创新的科学态度和解决实际
  问题的能力,培养使用计算机网络知识和方法解决分布式系统领域相关实际问
  题的能力。&%
  {作业、实验、\\考试}\\
  % ------------------------------------------------------------------------------------------------------------------------------------------
\end{support}

\section{教学内容、教学要求及学时分配}

\subsection{理论教学}
% 说明:课程思政内容与课程简介中描述要一致;课程思政内容不是每一章或者
% 每次课程都要融入。

\begin{lecture}{XXlcc}%{colspec={XXlcc},hlines,vlines,row{1}={c,m}}
  % {按章节或按教学顺序\\1.1\\1.2\\课程思政:} &%
  % 了解……理解……掌握……&%
  % {1. 课堂讲授\\ 2. 多媒体演示\\ 3. 任务驱动\\ 4. 翻转课堂\\5. 归纳总
  %   结\\ 6. 小组研讨/报告\\ 7. 答疑与互动\\ ……\\ 以上方
  %   法根据教学选择}&&\\
  % --------------------------------------------------------------------------------------------------------------------------------------------------------------------
  课程教学内容& 教学要求& 教学设计& {推荐\\学时}& {支撑课程\\目标}\\
  % --------------------------------------------------------------------------------------------------------------------------------------------------------------------
  {%课程教学内容
    1 分布式系统的概念与模型\\
    1.1 分布式系统的应用实例\\
    1.2 体系结构\\
    1.3 基础模型\\
    课程思政:理想信念教育}&%
  {%教学要求
    了解分布式系统发展的历史和常识,理解分布式系统的概念,并了解其
    基本工作原理。\\
    通过对我国当前在分布式系统研发领域的成果进行介绍,帮助学生建立民族
    自信心,激发学生树立中华民族复兴的志向。建立自信,激发志向/立足专业,勇攀高峰。}&%
  {%教学设计
    1. 课堂讲授\\
    2. 多媒体演示\\
    3. 答疑与互动}&%
  6&%推荐学时
  {指标点\\11.2}\\%支撑课程目标
  % --------------------------------------------------------------------------------------------------------------------------------------------------------------------
  {%
    2 TCP/IP网络基础\\
    2.1 网络原理\\
    2.2 因特网协议\\
    2.3 网络实例分析\\
    课程思政:工匠精神}&%
  {%
    理解互联网的工作原理和网络通信模型。\\
    结合本章的学习,将工匠精神逐步渗透到教学中,培养学生认真踏实的学习
  态度,充分发挥课程育人的功能。}&%
  {%
    1. 课堂讲授\\
    2. 多媒体演示\\
    3. 任务驱动\\
    4. 答疑与互动\\
    5. 归纳总结}&%
  6&{指标点\\11.2\\12.2} \\
  % --------------------------------------------------------------------------------------------------------------------------------------------------------------------
  {%
    3 进程间通信\\
    3.1 网络协议的编程接口\\
    3.2 C/S通信模型\\
    3.3 实例研究:Unix系统的进程间通信\\
    课程思政:可持续发展}&%
  {%
    了解进程间通信的基本原理,熟悉常用的网络编程接口函数。\\
    结合编程技术和网络技术,让学生意识到分布式系统开发的可持续性。}&%
  {%
    1. 课堂讲授\\
    2. 多媒体演示\\
    3. 任务驱动\\
    4. 答疑与互动\\
    5. 归纳总结}&%
  6&{指标点\\11.2\\12.2} \\
  % --------------------------------------------------------------------------------------------------------------------------------------------------------------------
  {%
    4 分布式对象和远程调用\\
    4.1 分布式对象间的通信\\
    4.2 远程过程调用\\
    4.3 事件和通知\\
    4.4 实例研究:Java RMI\\
    课程思政:工匠精神}&%
  {%
    了解分布式对象和远程调用的基本方法。\\
    结合编程技术和网络技术,让学生意识到分布式系统开发的可持续性。    
  }&%
  {%
    1. 课堂讲授\\
    2. 多媒体演示\\
    3. 任务驱动\\
    4. 答疑与互动\\
    5. 归纳总结}&%
  6&{指标点\\11.2\\12.2}\\
  % ------------------------------------------------------------------------------------------------------------------------------------------------------------------  
  % 课程教学内容& 教学要求& 教学设计& 推荐学时& 支撑课程目标\\
\end{lecture}

\subsection{教学方法}
% 以下教学方法,请根据实际教学,说明教学方法,以及采用该教学方法达到什么目的。

\begin{enumerate}
\item 课堂讲授与讨论:本课程的理论部分,主要采取课堂讲授为主,将课程中
  所涉及到的背景、概念、思想、方法以深入浅出的语言介绍给学生,并鼓励学
  生参与互动讨论,鼓励学生提问。
\item 多媒体演示:以图片(框图、流程图、时序图、程序示例等)为主,一图
  胜千言,围绕图片、示例展开解说,事半而功倍。
\item 任务驱动:在教学过程中,结合教学内容,适时地给学生布置一些小任务,
  以提高学生的教学参与度,使学生在完成小任务的过程中,加强对教学内容的
  理解。
\item 答疑互动:在教学过程中,鼓励学生提问,同时适时地向学生提问,以提
  高学生的参与度,启发学生思考,加强学生对教学内容的理解。
\end{enumerate}

\subsection{实验教学}%(如果没有课内实验,本部分内容删除)
% 实验类型为:演示型、验证型、综合型、设计型、创新型。
% 实验要求为:必做、选做。

实验教学是本课程中的重要实践环节,目的是培养学生的动手能力,让学生尽快
熟悉网络编程的工具,同时加深学生对网络概念与原理的理解。
本课程的实验教学为非独立设课,具体要求如下。

\begin{lab}{lXcccc}%{colspec={lXcccc},hlines,vlines,row{1}={c,m}}
  % --------------------------------------------------------------------------------------------------------------------------------------------------------------------  
  实验项目名称&教学要求&{实验\\学时}&{每组\\人数}&{实验\\类型}&{实验\\要求}\\
  % --------------------------------------------------------------------------------------------------------------------------------------------------------------------  
  {Linux平台上常用网络命令及开发环境介绍}&%
  {能够使用基本网络命令解决常见的网络问题,并能完成基本的Java编程}&%
  6&1&验证型&必做\\
  % --------------------------------------------------------------------------------------------------------------------------------------------------------------------  
  {Java网络编程入门}&%
  {进一步熟悉Linux平台上的Java开发环境,并尝试基本的网络接口函数}&%
  6&1&综合型&必做\\
  % --------------------------------------------------------------------------------------------------------------------------------------------------------------------  
  {Linux平台的进程间通信}&%
  {能够调用相关syscall编写简单的进程间通信程序}&%
  6&1&综合型&必做\\
  % --------------------------------------------------------------------------------------------------------------------------------------------------------------------  
  {Linux平台的网络编程}&%
  {理解并尝试Linux平台的套接字编程函数接口}&%
  6&1&验证型&必做\\
  % --------------------------------------------------------------------------------------------------------------------------------------------------------------------  
\end{lab}

\section{课程的考核环节}

\subsection{成绩评定法}
% 说明:以下考核项目、考核项目百分比根据课程评价依据而定。

\begin{itemize}
\item \(期末总成绩=平时成绩(25\%)+实验成绩(25\%)+期末卷面成绩(50\%)\)
\item \(平时成绩=考勤成绩(10\%)+课堂表现(40\%)+作业成绩(50\%)\)
\item \(实验成绩 = 实验过程成绩(50\%)+ 实验报告成绩(50\%)\)
\item \(期末卷面成绩 = 期末考试卷面成绩\)
\end{itemize}

\section{参考教材和资料}
% [1] 《xxxxxxxxxx》(第 x 版)[M]. xx:xxxxx 出版社, xxx.
% [2] 《xxxxxxxxxx》(第 x 版)[M]. xx:xxxxx 出版社, xxx.
% [3] 《xxxxxxxxxx》(第 x 版)[M]. xx:xxxxx 出版社, xxx.

% \begin{refsection}
%   \nocite{tanenbaum2011computer,fall2011tcp,kurose2013computer,%
%     bautts2005linux,hunt2002tcp,hall2009beej}
%   \printbibliography[heading=none]{}
% \end{refsection}

\booklist{}

\vfill
\begin{flushright}
  执笔人签字:\includegraphics[width=25mm]{signature}\qquad 审稿人签字:\makebox[2cm][c]{}\\
  主管教学院长签字:\makebox[2cm][c]{}
\end{flushright}

\end{document}
%%% Local Variables:
%%% mode: latex
%%% TeX-master: t
%%% End:
