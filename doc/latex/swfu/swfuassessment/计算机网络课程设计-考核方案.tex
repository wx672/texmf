\documentclass{swfuassessment}

\addbibresource{syllabus.bib}

\title{计算机网络课程设计}

\swfusetup{%
  % Dept=计算机科学与技术,%
  % Teacher=王晓林,%
  % Cls=必修课,%
  % Type=专业基础课,%
  % Sbj=,%
  Hours=18,%
  Book=xxr2017,%
}

\begin{document}

\headone{}

\section{考核目的}

课程考核目标及能力要求具体如下:通过本课程设计,使学生在已有的计算机
知识的基础上,对计算机网络从整体上有清晰全面的系统了解,对当前计算机网
络的主要种类和常用的网络协议有准确清晰的概念;对Linux平台上的网络管理
工具有较好的运用能力。

\section{考核对象}

计算机科学与技术2023级本科班

\section{考核时间和地点}

2024年12月16、17日,经管楼219机房

\section{考核过程}

完成考试网站上所布置的任务,并提交实习报告。

\begin{itemize}
\item \url{https://cs6.swfu.edu.cn/moodle/mod/assign/view.php?id=830}
\end{itemize}

\section{成绩构成要素及评分标准}

\begin{enumerate}
\item 课程成绩构成及比例:
  \begin{itemize}
  \item \(期末总成绩 = 平时成绩(50\%)+ 报告成绩(50\%)\)
  \end{itemize}
\item 各构成基本要素:
  \begin{itemize}
  \item \(平时成绩 = 考勤成绩(50\%)+ 实习表现(50\%)\)
  \item \(报告成绩 = 实习报告成绩\)
  \end{itemize}
\item 各基本要素评分标准:
  \begin{itemize}
  \item 缺勤1次扣除考勤成绩15\%,缺勤2次以上考勤成绩为0;
  \item 课堂听课认真,积极参与课堂教学问答者,课堂表现为满分;课堂听课
    不认真,随意交头接耳影响教学者,课堂表现不及格;
  \item 实习报告格式规范,内容详实。
  \end{itemize}
\end{enumerate}
\vfill
\begin{flushright}
  系主任签字:\hspace{4cm}学院负责人签字:\hspace*{2cm}\par
  \bigskip
  2024年11月1日
\end{flushright}

\headtwo{}

\section{过程性考核}

过程性考核由考勤、课堂表现、平时作业三个方面组成,占比及标
准如下表所示。

\begin{assessment}{ccXXcX}
  序号&{考核\\项目}&{考核要求}&{评分标准}&{占比\\(\%)}&{证明材料\\(文字或截图)}\\
  1&考勤&%
  不得无假条、无理由缺勤&%
  缺勤1次扣除考勤成绩15\%,缺勤3次以上考勤成绩为0;&%
  10\%&考勤记录表\\
  2&{课堂\\表现}&%
  认真听讲,积极参与课堂问答&%
  课堂听课认真,积极参与课堂教学问答者,课堂表现为满分;课堂听课
  不认真,随意交头接耳影响教学者,课堂表现不及格;&%
  40\%&考勤记录表\\
  3&{平时\\作业}&%
  认真完成作业&%
  作业成绩依作业的数量和质量而定;&%
  50\%&教学网站截图\\
  4&{阶段性\\测试}&无&无&&\\
  5&{报告或\\大作业}&无&无&&\\
  6&{程序}&无&无&&\\
  7&{其它}&无&无&&\\  
\end{assessment}

\end{document}

%%% Local Variables:
%%% mode: latex
%%% TeX-master: t
%%% End:
