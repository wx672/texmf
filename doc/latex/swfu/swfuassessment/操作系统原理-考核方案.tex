\documentclass{swfuassessment}

\addbibresource{syllabus.bib}

\title{操作系统原理}

\swfusetup{%
  % Dept=计算机科学与技术,%
  % Teacher=王晓林,%
  % Cls=必修课,%
  % Type=专业基础课,%
  % Sbj=,%
  Hours=64,%
  Book=tangxiaodan21,%
}

\begin{document}

\headone{}

\section{考核目的}

课程考核目标及能力要求具体如下:通过本课程的学习,使学生在已有的计算机
知识的基础上,对操作系统从整体上有清晰全面的系统了解,对当前操作系统的
主要种类和常用的操作系统有准确清晰的概念;学习进程、线程、内存管理、文
件系统的基本概念和各模块之间的联系;掌握操作系统设计的模块划分各模块的
基本服务功能;具备使用计算机操作系统知识知识解决相关实际问题的能力。

\section{考核对象}

计算机科学与技术2021级

\section{考核时间和地点}

2023年12月25日,经管楼219机房

\section{考核过程}

机试,完成考试网站上所布置的考题。

\begin{itemize}
\item \url{https://cs6.swfu.edu.cn/moodle/mod/quiz/view.php?id=763}
\end{itemize}

\section{成绩构成要素及评分标准}

\begin{enumerate}
\item 课程成绩构成及比例:
  \begin{itemize}
  \item \(期末总成绩 = 平时成绩(25\%)+ 实验成绩(25\%)+ 期末卷面成绩(50\%)\)
  \end{itemize}
\item 各构成基本要素:
  \begin{itemize}
  \item \(平时成绩 = 考勤成绩(10\%)+ 课堂表现(40\%)+ 作业成绩(50\%)\)
  \item \(实验成绩 = 实验过程成绩(50\%)+ 实验报告成绩(50\%)\)
  \item \(期末卷面成绩 = 期末考试卷面成绩\)
  \end{itemize}
\item 各基本要素评分标准:
  \begin{itemize}
  \item 缺勤1次扣除考勤成绩15\%,缺勤3次以上考勤成绩为0;
  \item 课堂听课认真,积极参与课堂教学问答者,课堂表现为满分;课堂听课
    不认真,随意交头接耳影响教学者,课堂表现不及格;
  \item 作业成绩依作业的数量和质量而定;
  \item 期末卷面成绩依《计算机网络·参考答案及评分标准》而定。
  \end{itemize}
\end{enumerate}
\vfill
\begin{flushright}
  系主任签字:\hspace{4cm}学院负责人签字:\hspace*{2cm}\par
  \bigskip
  2022年6月1日
\end{flushright}

\headtwo{}

\section{过程性考核}

过程性考核由考勤、课堂表现、平时作业三个方面组成,占比及标
准如下表所示。

\begin{assessment}{ccXXcX}
  序号&{考核\\项目}&{考核要求}&{评分标准}&{占比\\(\%)}&{证明材料\\(文字或截图)}\\
  1&考勤&%
  不得无假条、无理由缺勤&%
  缺勤1次扣除考勤成绩15\%,缺勤3次以上考勤成绩为0;&%
  10\%&考勤记录表\\
  2&{课堂\\表现}&%
  认真听讲,积极参与课堂问答&%
  课堂听课认真,积极参与课堂教学问答者,课堂表现为满分;课堂听课
  不认真,随意交头接耳影响教学者,课堂表现不及格;&%
  40\%&考勤记录表\\
  3&{平时\\作业}&%
  认真完成作业&%
  作业成绩依作业的数量和质量而定;&%
  50\%&教学网站截图\\
  4&{阶段性\\测试}&无&无&&\\
  5&{报告或\\大作业}&无&无&&\\
  6&{程序}&无&无&&\\
  7&{其它}&无&无&&\\  
\end{assessment}

\end{document}

%%% Local Variables:
%%% mode: latex
%%% TeX-master: t
%%% End:
