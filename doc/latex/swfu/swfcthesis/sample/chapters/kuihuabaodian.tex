\chapter{葵花宝典}
葵花寶典\footnote{\href{http://zh.wikipedia.org/wiki/\%E8\%91\%B5\%E8\%8A\%B1\%E5\%AE\%9D\%E5\%85\%B8}{葵花宝典}},是金庸武俠小說《笑傲江湖》中的禁断的武功秘笈奥義,書中對其來歷著墨不多,相傳作者為前朝\footnote{根據前後文推測前朝为南宋或元朝;金庸言明笑傲江湖不特定某一朝代,根据人物对话及武功推测时间在宋元之后,多数人认为是明朝}
太監,為何太監武功如此高強、卻身居大內,不得而知。此功是以快為主,令人沒有還擊的機會,已將別人打敗\cite{khbd}。

\section{由來}

《葵花寶典》之後流傳到福建少林寺,當時正好華山派门人岳肅與蔡子峰拜訪,趁機各偷抄一部分,被紅葉禪師發覺,認為此害人之物不得留世,於是焚毀。

岳、蔡二人返回華山後,彼此把各自抄寫部分拿出比對,竟然不合,於是互相怀疑以至兄弟反目。

从此二人文爭武鬥,激起華山劍宗與氣宗之爭。小說中風清揚屬劍宗、岳不群屬氣宗。

後日月神教大舉攻入華山派,為的就是奪取《葵花寶典》殘本,激鬥後「日月神教十長老」戰死於華山派,但寶典亦被日月神教奪去,輾轉由東方不敗習得。

上述這一段,乃令狐沖與少林寺方證大師、武當派沖虛道長密會時,由方證與沖虛訴說。

\section{分拆出的武功}

另一段,福建少林寺的和尚渡元奉命前往華山讨要宝典,岳、蔡二人直承不諱,兩人並向渡元禪師請教寶典裡面的武學,認為渡元禪師為紅葉禪師高徒,必有蒙紅葉禪師傳授寶典裡面武學。而渡元靠自身領悟力隨意解釋一番。憑著小部分的記憶,將自己領悟到的記下写于袈裟之上,後來憑此自創出七十二路「辟邪劍法」,後來也不回福建少林寺,還俗並自稱為林遠圖,開設鏢局。該劍谱才是笑傲江湖奪取之主要典籍。

據方證大師所言,林遠圖所悟遠較日月神教奪搶而去之華山派筆錄殘本為多。

\section{修練時的重點}

練習葵花寶典之前,在此典的第一頁已經註明「欲練神功,引刀自宮。煉丹服藥,內外皆通。」(另一說是「欲練此功,必先自宮。不丹不藥,內外皆通」)這句子,意思是修練前必先自宮,否則會『欲火如焚,登時走火入魔,僵癱而死』。東方不敗曾曰:「我初當教主,那可意氣風發了,說什麼文成武德,中興聖教,當真是不要臉的胡吹法螺。直到后來修習《葵花寶典》,才慢慢悟到了人生妙諦。其后勤修內功,數年之后,終于明白了天人化生、萬物滋長的要道。」
另外,岳不群跟林平之都已自宮,但根據金庸原著,岳不群與林平之都不是練《葵花寶典》,因為寶典在日月神教東方不敗之手,兩人練的是由林遠圖改編過的「辟邪劍法」。

\section{其他}

葵花寶典實在的紀載僅一次,亦即在黑木崖上令狐冲、任我行、向問天合攻東方不敗,但僅打成平手,且任我行失去一眼。

在對峙前,東方不敗曾以繡花針攻擊令狐冲,令湖冲以獨孤九劍對應,直擊東方不敗要害的兩敗俱傷打法,逼退東方不敗。東方不敗曾稱讚:「好高的劍法!」

%%% Local Variables: 
%%% mode: latex
%%% TeX-master: "../sample"
%%% End: 
