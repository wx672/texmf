\chapter{空明拳}
空明拳\footnote{参见维基百科 - \href{http://zh.wikipedia.org/wiki/\%E7\%A9\%BA\%E6\%98\%8E\%E6\%8B\%B3}{空明拳}}是金庸小說《射鵰英雄傳》、《神鵰俠侶》、《倚天屠龍記》中出現的武功。

空明拳是周伯通被黃藥師困於桃花島時,在所住山洞裡自創的武功。空明拳十六字訣:空朦洞鬆、風通容夢、沖窮中弄、童庸弓蟲。空明拳共七十二路,第一路空碗盛飯、第二路空屋住人、深藏若虛(《神鵰俠侶三六回-獻禮祝壽,耶律齊對上何師我》)、五十四路妙手空空(《倚天屠龍記第一回-天涯思君不可忘,郭襄對上無色》),均為道家的基本概念化成。他在山洞裡和郭靖結為兄弟後,將空明拳傳授給郭靖,之後亦傳給耶律齊。

%%% Local Variables: 
%%% mode: latex
%%% TeX-master: "../sample"
%%% End: 
