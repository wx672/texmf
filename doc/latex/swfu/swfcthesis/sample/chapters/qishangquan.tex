\chapter{七傷拳}
七傷拳\footnote{参见维基百科 - \href{http://zh.wikipedia.org/wiki/\%E4\%B8\%83\%E5\%82\%B7\%E6\%8B\%B3}{七傷拳}}為金庸武俠小說《倚天屠龍記》中崆峒派的武功,當年崆峒派開山祖師木靈子曾持之威揚天下,但後來崆峒五老內功不足卻強練,因此人人暗伏內傷,反倒從他們手裡奪去秘笈的謝遜更將這門拳法發揮的淋漓盡致,而張無忌則輾轉由謝遜處知悉七傷拳要訣,在維護明教力鬥六大派高手時,在崆峒五老面前炫耀七傷拳功。\footnote{出自倚天屠龍記第二十一章 排難解紛當六強}

\section{介紹}

七傷拳的總訣是四句似歌非歌、似詩非詩的拳訣:「五行之氣調陰陽,損心傷肺摧肝腸,藏離精失意恍惚,三焦齊逆兮魂魄飛揚!」武學宗旨在於先傷己再傷人,一練七傷損及內臟跟陰陽二氣,內力不足者將受創甚深,崆峒五老人人皆練,都暗隱內傷。

由於七傷拳功施展開來威勢顯赫,和成昆的「霹靂拳」等武技外觀類似,但內醞七種迥異拳勁,剛猛、陰柔、剛中有柔、柔中有剛、橫出、直送、內縮兼備。為此謝遜闖入崆峒山青陽觀劫奪拳譜,本來謝遜勢必不敵崆峒五老聯手,但成昆一心挑撥六大派和明教不和,暗地出手相救,用混元功打傷唐文亮、常敬之,使謝遜奪得拳譜,但謝遜馬上著手修練後,傷了心脈導致有時狂性大發\footnote{出自倚天屠龍記第八章 窮發十載泛歸航}。

在謝遜收張無忌為義子後,把七傷拳傳給了張無忌,並在維護明教力鬥崆峒派高手時,張無忌用九陽神功催動七傷拳技壓崆峒五老,但也用九陽真氣暗助宗維俠解除部分內傷,以德服人。由於張無忌的九陽神功極為渾厚,因此他使用七傷拳時完全沒受半點反撲傷害,威力亦更勝崆峒五老跟謝遜。

%%% Local Variables: 
%%% mode: latex
%%% TeX-master: "../sample"
%%% End: 
