\chapter{龍爪擒拿手}
龍爪擒拿手
\footnote{参见维基百科 - \href{http://zh.wikipedia.org/wiki/\%E9\%BE\%8D\%E7\%88\%AA\%E6\%93\%92\%E6\%8B\%BF\%E6\%89\%8B}{
    龍爪擒拿手}}為金庸武俠小說《倚天屠龍記》中之武功,乃少林七十二絕技之一,是空性在光明頂激戰張無忌
時所用的武功。

\section{簡介}

龍爪擒拿手是少林七十二絕技之一,乃少林派四大神僧中空性的絕技,雖說四大神僧全都練就,但以空性造詣最高。

龍爪手全套只有三十六招,要旨端在凌厲狠辣,不求變化繁多。空性中年之時曾數逢大敵,僅要使出龍爪手便立占上風,通常用到十二招便即取勝,從第十三招起,只是空性自己平時練習之用。

被張無忌評說是沒有破綻的不敗武功,藉由身法拉開兩、三尺距離後,仔細端詳偷學模仿,最後用一模一樣的龍爪手破解空性的龍爪手,兩人英雄識英雄反倒結交。

在少林派離開光明頂後,空性一路被蒙古軍伏撃時,因提早發現對方使用時香軟筋散暗算而起衝突,空性以「龍爪
擒拿手」與汝陽王府旗下西域少林金剛門僧人「阿三」的「大力金剛指」大戰時,指力鬥指力,空性不敵被殺\footnote{出自倚天屠龍記第二十四章 太极初傳柔克剛}。

\section{招式}

書中有記載的招式,記載如下\footnote{出自倚天屠龍記第二十一章 排難解紛當六強}:

\begin{itemize}
\item 第八招「拿雲式」:左手虛探,右手挾著一股勁風,直拿人左肩缺盆穴。
\item 第十二招「搶珠式」:此招和拿雲式外表雖同,但方位卻異,雙手一樣自上而下同抓,卻是抓人左右太陽穴。
\item 第十七招「撈月式」:虛拿人後腦風府穴。
\item 捕風式、捉影式、撫琴式、鼓瑟式、批亢式、擣虛式:套路不明,八式連環不絕,便如一招中的八個變化一般,快捷無比。
\item 第三十五招「抱殘式」、第三十六招「守缺式」:龍爪手最後兩招,至剛生柔,看似破綻百出,實則暗藏陷阱。
\end{itemize}

%%% Local Variables: 
%%% mode: latex
%%% TeX-master: "../sample"
%%% End: 
