\chapter{北冥神功}
北冥神功\footnote{参见维基百科 - \href{http://zh.wikipedia.org/wiki/\%E5\%8C\%97\%E5\%86\%A5\%E7\%A5\%9E\%E5\%8A\%9F}{北冥神功}}為金庸武俠小說《天龍八部》的虛構內功,與化功大法皆屬逍遙派,旨在吸取別人之內力收為己用,與化
功大法的用意化去別人內力頗不相同。全書中會運使人物包括無崖子、段譽、虛竹,其中以段譽受益最高。到了
《笑傲江湖》中,北冥神功從大理段氏流傳並且與星宿派化功大法融合成為吸星大法。

\section{概述}

北冥神功為逍遙派絕學之一,原為無崖子獨得,後由李秋水所書修行圖示,段譽得到修行帛卷,但最後亦毀於段譽之手。

「北冥」之詞出自莊子〈逍遙游〉:「窮發之北有冥海者,天池也。有魚焉,其廣數千里,未有知其修也。」,武功原理為「百川匯海,大海之水以容百川而得」,意即吸取別人的內力而收為己用。全書共分三十六幅圖像,其中「手太陰肺經」為入門第一課,其中修練方式又與世俗武功反其道而行,順序為:

少商→魚際→大淵→經渠→列缺→孔最→尺澤→俠白→天府→雲門→中府
北冥神功修煉後吸納來的內力,不同於經吸星大法吸納的各種內力,仍帶有原本內力的性質,而是轉化成一種統一
性質的內力,稱北冥真氣。 《天龍八部》一書中,段譽只修煉過「手太陰肺經」和「任脈」兩幅圖,並無法自運
北冥神功。隨著故事的進行,記載神功圖像的法門因故遭毀,後流傳的北冥神功也僅剩段譽所學,故推測流傳至
《笑傲江湖》的北冥神功應為段譽所學僅存的北冥神功。 虛竹曾受無崖子傳北冥真氣,並且曾由天山童姥指點運
勁法門,但由於無崖子並未傳授北冥神功,故虛竹僅身具北冥真氣而無法藉吸納來修煉北冥真氣。

\section{比較}

北冥神功、化功大法、吸星大法三者都為吸取別人內力的武功,三者在一定程度上有相似之處。 三者之中,化功
大法有別於另兩者:前者是以劇毒令人手腳麻痺,便似內力盡失,實只須解去毒性,則能重獲內力;另兩者則確實
吸去內力,收為己用。缺點方面,吸星大法會有所謂內力反噬現象;也就是所吸納的諸般異種真氣,無法融合而互
相沖激,而且每吸一次發作的間隔越近、痛楚越增。吸星大法雖源於另兩者,但其缺點較多得自化功大法,故推斷
化功大法應也有相當程度的缺點;北冥神功的缺點在於,當遇上內力更強的對手時,若強行吸納會有經脈斷裂之虞,
是為「然敵之內力若勝於我,則海水倒灌而入江河,凶險莫甚,慎之,慎之。」\footnote{天龍八部 第二章 玉璧
  月華明}

%%% Local Variables: 
%%% mode: latex
%%% TeX-master: "../sample"
%%% End: 
