\chapter{关于金庸}
\label{sec:appendixname}

\section{金庸简介}

金庸\footnote{参见维基百科-\href{http://zh.wikipedia.org/wiki/\%E9\%87\%91\%E5\%BA\%B8}{金
      庸}},大紫荊勳賢,OBE,原名查良鏞,(Louis Cha Leung Yung,1924年2月6日-)浙江海寧人,東吳大
  学法學士、英國劍橋大學歷史碩士、博士。查良鏞在1948年移居香港,以筆名金庸著作多部膾炙人口的武俠小說,
  是华人界最知名的武俠小說作家之一。金庸亦是香港新闻、文艺界的杰出创业者及评论家,以及著名的社会活动
  家。金庸、古龍與梁羽生普遍被認為是新派武侠小说的代表作家,被誉为武侠小说作家的「泰斗」,更有「金迷」
  们尊称其为「金大侠」或「查大俠」,亦被喻為「香港四大才子」之一\cite{jinyongcn}。
\begin{itemize}
\item 傳說金庸對古典樂十分喜好,判別能力超強。可以隨便聽過一段樂句,就告訴你這是哪位作曲家的哪首作品。
\item 金庸愛車,跑車为尤。
\item 接受訪談時,金庸認為自己像張無忌(《倚天屠龍記》),妻子像夏青青(《碧血劍》)。
\item 金庸在《鹿鼎記》后記中,表示自己最喜歡的作品是《倚天屠龍記》,《笑傲江湖》,《神鵰俠侶》和《飛狐外
  傳》。
\item 金庸在《倚天屠龍記》后記中,表示自己最愛的女性人物是小昭(《倚天屠龍記》)。
\item 曾开除金庸的國立政治大學(其前身即中央政治學校)于2007年5月19日,授予金庸榮譽博士學位。
\item 金庸於香港大學設立“查良鏞學術基金”,並擔任主席,主力邀講各國學者定期舉行學術講座和研討會。
\item 香港大學於2009年2月27日成立國際金庸研究會。研究會並非註冊社團,由金庸及前港大中文系主任單周堯教授擔任顧問,創會會長為李思齊教授及伍懷璞教授,現任會長為港大文學院黎活仁副教授\cite{jinyongcn}。
\end{itemize}

\clearpage
\section{金庸年表}\footnotetext{参见金庸图书馆 -
  \href{http://jinyong.ylib.com.tw/practice/about/about4.htm}{金庸年表}}
\begin{table}[h]
  \centering
  \caption{金庸年表}\label{tab:timeline}
  \vspace{-4ex}
  \tabulinesep=1mm
  \begin{tabu}to \textwidth {X[c]X[c]X[8,l]}\\\hline
    西元&年龄&\centering{大事记}\\\tabucline-
    1955&31&以「金庸」為筆名,創作第一部武俠小說《書劍恩仇錄》,在《新晚報》連載一年,奠定武俠文學基業。\\
    % 1956&32&《碧血劍》開始在《香港商報》連載。並與梁羽生、百劍堂主在《大公報》開闢「三劍樓隨筆」專欄。\\
    % 1957&33&進入長城電影公司。寫《射鵰英雄傳》連載於《香港商報》。\\
    % 1958&34&與程步高合導電影《有女懷春》。\\
    % 1959&35&與胡小峰合導電影《王老虎搶親》。創辦《明報》,《神鵰俠侶》開始在《明報》創刊號連載。《雪
    % 山飛狐》連載於《新晚報》。 \\
    % 1960&36&為《武俠與歷史》雜誌撰寫《飛狐外傳》。\\
    % 1961&37&《倚天屠龍記》、《鴛鴦刀》、《白馬嘯西風》開始在《明報》連載。\\
    % 1962&38&《明報》因報導「逃亡潮」而聲名大噪,發行量遽增。\\
    % 1963&39&為《東南亞周刊》撰寫《連城訣》。《天龍八部》開始在《明報》連載。\\
    1964&40&發表〈寧要褲子,不要核彈〉社評。與《大公報》展開一系列筆戰。\\
    1965&41&創辦《明報月刊》,創作《俠客行》。\\
    1966&42&對「文革」做一系列分析。\\
    1967&43&香港爆發「六七暴動」,《明報》成為左派分子重點襲擊目標。在馬來西亞及新加坡創辦《新明日
    報》。在香港創辦《明報周刊》。創作《笑傲江湖》。 \\
    1969&45&創作、發表巔峰之作《鹿鼎記》。\\
    1970&46&寫《越女劍》。開始修訂全部武俠小說作品。\\
    1972&48&《鹿鼎記》連載完畢,宣布就此封筆不寫武俠小說。\\
    1973&49&以《明報》記者身分赴台訪問十天,會見嚴家淦、蔣經國等,之後於《明報》連載〈在台所見.所聞.
    所思〉。 \\
    1979&55&參加台北舉行之「建國會」,與丁中江共同為小組討論會之主席。正式授權台灣遠景出版社出版《金
    庸作品集》。\\ 
    1980&56&廣州《武林》雜誌連載《射鵰英雄傳》,金庸武俠小說正式進入大陸。十五部卅六冊《金庸作品集》
    全部修訂完畢,前後花了十年時間。\\ 
    1981&57&與妻子兒女回大陸訪問,會見鄧小平,並遊歷十三個城市。獲頒英國政府O.B.E.勳銜。\\\hline
  \end{tabu}
\end{table}

\clearpage
\section{金庸武侠世界编年表}
\footnotetext{参见:\href{http://www.taihainet.com/lifeid/culture/jianghu/201005/535567.html}{金庸武
    侠世界编年表}}
\footnotetext{本年表未尽囊括金庸小说中所有人物及事件,属于不完整版。另:本文纯属网友娱乐。} 

\begin{longtabu}to \textwidth {rX[3.4,l]}
  \caption{金庸武侠世界编年表\label{tab:historyline}}\\\tabucline-
  {\centering 年代}&\centering{武林大事}\\\tabucline-\endfirsthead
  \caption[]{金庸武侠世界编年表(续)}\\\hline\endhead
  公元前476年&西施送入吴国;范蠡遇到阿青,阿青传越国剑士剑法。\\
  公元前473年&越国灭掉吴国;范蠡与西施隐居,阿青离去。 \\
  526年&南北朝时期,印度高僧菩提达摩来到中国,在嵩山少林寺面壁九年,创立中国禅宗。 \\
  隋朝末年&李靖将《易筋经》中的武学奥秘,尽数领悟。 \\
  694年&明教传至中土。 \\
  唐朝末叶&嘉兴剑术名家改良越女剑法。 \\
  877年&丐帮建立。 \\
  936年—946年&少林寺法慧禅师练成了一指禅。 \\
  1030年&慕容博出生。 \\
  1047年&慕容博伤黄眉僧。 \\
  1051年&扫地僧到少林寺。 \\
  1060年&萧峰出生。 \\
  1061年&雁门关外乱石谷大战。 \\
  1062年&萧远山到少林偷研武功。 \\
  1063年&丁春秋暗算师门。 \\
  1064年&慕容复出生。 \\
  1065年&波斯“山中老人”霍山自创“乾坤大挪移”。 \\
  1069年&虚竹出生。 \\
  1071年&段誉出生。 \\
  1072年&慕容博伤崔百泉。 \\
  1074年&阿朱出生。 \\
  1077年&慕容博诈死。 \\
  1083年&萧峰接任丐帮帮主。 \\
  1090年&鸠摩智单挑天龙寺六僧。 \\
  1091年&萧峰离开丐帮。 \\
  1092年&聚贤庄大战。 \\
  1093年&无崖子去世;虚竹接任逍遥派掌门,同年被西夏招为驸马;少林寺门外混战;扫地僧讲经说法。 \\
  1094年&大理段正明禅位侄子段誉;萧峰自尽。 \\
  1110年&黄裳雕版印行万寿道藏。 \\
  1112年&王重阳出生。 \\
  1120年&黄裳向明教的高手挑战。 \\
  1127年前后&少林寺灵兴禅师花了三十九年练成了一指禅。 \\
  1140年&独孤求败创独孤九剑第九式破气式。 \\
  1148年&丘处机出生。 \\
  1158年&段智兴出生。 \\
  1162年&柯镇恶出生(江南七怪之首)。 \\
  1163年&周伯通出生。 \\
  1165年&黄裳完成九阴真经。洪七公出生。 \\
  1169年&欧阳锋出生。 \\
  1170年&独孤求败郁寂而终。 \\
  1171年&黄药师出生。 \\
  1173年&曲灵风出生。 \\
  1178年&裘千仞出生。 \\
  1183年&金轮法王出生。 \\
  1184年&朱子柳出生。 \\
  1186年&陈玄风 欧阳克出生。 \\
  1190年&火工头陀大开杀戒。 \\
  1196年&王重阳再入古墓,于棺上刻下部分九阴真经,并留“****,欲胜全真;重阳一生,不弱于人” 的文字,
  后某日无名僧与王重阳斗酒参阅九阴创九阳。\\
  &宦官创葵花宝典。 \\ 
  1202年&第一次华山论剑。 \\
  1203年&重阳真人拜访段皇爷,废了欧阳峰20年蛤蟆功;\\
  &王重阳仙逝。 \\
  1206年&裘千仞伤瑛姑幼子。 \\
  1207年&郭靖出生。 \\
  1220年&小龙女出生。 \\
  1223年&穆念慈比武招亲。 \\
  1225年&嘉兴烟雨楼比武;\\
  &杨康去世。 \\
  1226年&杨过出生。 \\
  1227年&第二次华山论剑;\\
  &成吉思汗去世。 \\
  1242年&洪七公、欧阳峰去世。 \\
  1243年&杨过小龙女双剑合壁初试锋芒挫败金轮法王;\\
  &杨过学弹指神通。 \\
  1244年&郭襄出生;\\
  &杨过断臂,初入独孤求败剑冢。 \\
  1247年&张三丰出生。 \\
  1257年&明教石教主圣火令为丐帮所夺。 \\
  1259年&杨过飞石击毙蒙哥大汗;第三次华山论剑。 \\
  1262年&郭襄游少林。 \\
  1273年&一代大侠郭靖战死襄阳。 \\
  1296年&金毛狮王谢逊出生。 \\
  1317年&谢逊离开师父成昆,加入明教。 \\
  1318年&武当六弟子殷梨亭出生。 \\
  1323年&成昆杀谢逊一家。 \\
  1336年&张三丰九十大寿;谢逊及张翠山夫妇至**岛。 \\
  1337年&张无忌出生。 \\
  1338年&元兵剿灭袁州明教义军,常遇春、彭莹玉侥幸逃脱。 \\
  1339年&周芷若出生。 \\
  1340年&汝阳王女儿敏敏特穆尔出生,元帝封其“绍敏郡主”。 \\
  1341年&小昭出生。 \\
  1346年&张三丰百岁大寿;张翠山殷素素夫妇自尽。 \\
  1351年&张无忌得九阳神功。 \\
  1357年&六大派围攻光明顶;张无忌任明教教主;\\
  &张三丰首创太极神功。 \\
  1358年&小昭远走波斯。 \\
  1359年&少林屠狮英雄会;白眉鹰王殷天正去世。 \\
  1360年&张无忌隐退,光明左使杨逍继任为明教第三十五代教主。 \\
  1396年&武当派开山祖师张三丰仙逝,享年149岁。 \\
  1400年&莆田少林寺得葵花宝典。 \\
  1401年&岳肃蔡子峰偷录葵花宝典;渡元禅师习辟邪剑法。 \\
  1402年&华山派分气宗剑宗。 \\
  1406年&日月神教十长老破五岳剑派剑法。 \\
  1420年&日月教袭武当山,张三丰手抄一部《太极拳经》和青年时佩带的真武剑被抢。 \\
  1479年&华山派气宗剑宗之争。 \\
  1486年&任盈盈出生。 \\
  1493年&东方不败篡日月神教教主之位;任我行被囚地牢。 \\
  1503年&余沧海灭福威镖局。 \\
  1504年&令狐冲学得独孤九剑。 \\
  1505年&任我行重夺日月神教教主。 \\
  1506年&任我行去世。 \\
  1509年&令狐冲任盈盈喜结良缘。 \\
  1610年&金蛇郎君夏雪宜惨遭灭门之祸。 \\
  1612年&金蛇郎君夏雪宜得到云南五仙教的镇教“三宝”。 \\
  1623年&袁承志出生。 \\
  1643年&袁承志被推举为七省江湖首领,同年率众人挫毁西洋红夷大炮。 \\
  1644年&袁承志助阿九挫败成王与曹化淳篡位阴谋;李自成攻入北京,明亡;吴三桂降清;李岩夫妇自杀身亡。 \\
  1645年&袁承志、夏青青避居海外。 \\
  1655年&韦小宝出生。 \\
  1669年&康熙韦小宝擒鳌拜。 \\
  1670年&韦小宝出任天地会青木堂香主。 \\
  1698年&武当派掌门陆菲青出生。 \\
  1733年&陈家洛出生。 \\
  1753年腊月&苗人凤和胡一刀决战;胡斐出生。 \\
  1780年3月15日&苗人凤和胡斐决战。\\\hline
\end{longtabu}
%%% Local Variables: 
%%% mode: latex
%%% TeX-master: "../sample"
%%% End: 
