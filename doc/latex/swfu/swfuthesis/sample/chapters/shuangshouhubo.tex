\chapter{双手互搏}
雙手互搏\footnote{参见维基百科 -
  \href{http://zh.wikipedia.org/wiki/\%E5\%8F\%8C\%E6\%89\%8B\%E4\%BA\%92\%E6\%90\%8F}{雙手互
    搏}}(亦稱左右互搏)為金庸武俠小說《射鵰英雄傳》中「老頑童」周伯通\footnote{也就是《神雕俠侶》裡
  的「中頑童」}被「東邪」黃藥師困於桃花島時所創。

\section{概述}
據《神鵰俠侶》\cite{shendiao}原文: 其實這左右互搏之技,關鍵訣竅全在「分心二用」四字。凡是聰明智慧的人,心思繁複, 一件事沒想完,第二件事又湧上心頭。三國時曹子建七步成詩,五代間劉鄖用兵,一步百計,這等人要他學那左右互搏的功夫,便是要殺他的頭也學不會的。

關鍵訣竅全在「分心二用」,因此要習此門功夫,須做到心無雜念。根據周伯通的說法,若能左手畫方,右手畫圓,方能修習此法。《射鵰英雄傳》中世上僅有周伯通、郭靖能使用,《神鵰俠侶》中小龍女以養蜂術與周伯通交換雙手互搏之後便能一人使出「玉女素心劍法」,成為第三個懂得雙手互搏的人。周伯通,郭靖與小龍女皆是心思純樸之人,心無雜念,尤其小龍女自幼便學習寡欲,學習雙手互搏並非難事。

在倚天屠龍記中「崑崙三聖」何足道亦曾以左手凌厲攻敵、右手舒緩撫琴。儘管不及雙手分使兩般武功,卻已是射鵰三部曲中極少數能分心二用的人。

另外,碧血劍中,主角袁承志為圓溫青青一時口舌之快,於擊敗仙都派兩大弟子後,現學現賣,當場使出需兩人才有辦法施展的「兩儀劍法」,書中有云:「只見他雙劍舞了開來,左攻右守,右擊左拒,一招一式,果然與兩儀劍法毫無二致。劍招繁複,變化多端,洞玄和閔子華適才分別使出,人人都已親見,此時見他一人雙劍竟囊括仙都派二大弟子的劍招,盡皆相顧駭然。 」,可見袁承志亦是少數能分心二用之人。

還有《神鵰俠侶》的公孫止的「陰陽倒亂刃法」,能用刀使劍法,劍使刀法,都是一些能分心二用的高手。

%%% Local Variables: 
%%% mode: latex
%%% TeX-master: "../sample"
%%% End: 
