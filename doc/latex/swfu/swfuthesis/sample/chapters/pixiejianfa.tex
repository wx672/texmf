\chapter{辟邪劍法}

辟邪劍法\footnote{参见维基百科 - \href{http://zh.wikipedia.org/wiki/\%E8\%BE\%9F\%E9\%82\%AA\%E5\%8A\%8D\%E6\%B3\%95}{辟邪劍法}},金庸武俠小說《笑傲江湖》中的絶世劍法,載有劍法的《辟邪劍譜》,是小說中各人物所爭奪的武林秘笈,亦是引出整個故事的主要書籍\cite{pxjp}。

在令狐沖與少林方证大师、武當沖虛道长密會時,方證與沖虛告訴令狐沖有關劍法的來歷:當年福建少林寺的和尚渡元奉紅葉禪師之命前往華山讨回被華山派门人岳肅與蔡子峰偷錄的《葵花寶典》殘本,但蔡岳二人誤以為渡元禪師曾修習葵花寶典,反而藉機詢問寶典上的武學疑義,渡元一邊以自身武學基礎回應,一邊暗自記憶寶典內容。渡元靠過人記憶力,記下寶典殘本写于袈裟之上,并自創出七十二路《辟邪劍法》,後來也不回福建少林寺,還俗並自稱為林遠圖,開設鏢局,名震江湖。

其後劍譜傳至林震南夫婦時,由於不懂如何練就劍法,因此鏢局之名日墮,並且引來青城派覬覦,掌門人余滄海更將林家滅門以圖奪取劍譜;林家唯一倖存者林平之被華山派掌門岳不群救回並收為弟子。林震南死前向令狐沖留下遺言,使林平之得以尋回劍譜,結果卻落入岳不群手中。岳、林二人靠著劍譜練成驚人武功,最後岳不群與左冷禪在封禪台上決鬥,岳不群以真劍法對抗左冷禪的假劍法,在點瞎左冷禪雙目後取得勝利,成為五嶽派的掌門人,林平之亦得以此劍法報滅門之仇。由於辟邪劍法威名太盛,加上令狐沖得華山派劍宗遺老風清揚傳授獨孤九劍後武功大進,早期亦被岳、林二人懷疑其奪取辟邪劍譜。

與《葵花寶典》的修行方法一樣,練辟邪劍法者必先自宮。而根據曾親眼目睹該劍法的各人所言,劍法本身並無特別厲害之處,全仗自宮後所修練的內功,使劍法威力倍增。而令狐沖與岳不群對戰之時,終領悟到劍法的厲害在於一個「快」字,只要時間一久,劍招便會重複,破綻亦會隨之顯露,令狐沖亦憑此得以破解了辟邪劍法。

%%% Local Variables: 
%%% mode: latex
%%% TeX-master: "../sample"
%%% End: 
