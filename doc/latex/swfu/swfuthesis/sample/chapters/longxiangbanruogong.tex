\chapter{龍象般若掌}
龍象般若功
\footnote{参见维基百科 - \href{http://zh.wikipedia.org/wiki/\%E9\%BE\%8D\%E8\%B1\%A1\%E8\%88\%AC\%E8\%8B\%A5\%E6\%8E\%8C}{
    龍象般若功}}為金庸武俠小說《神雕俠侶》中之武功,乃金輪法王最強的武功,共分十三層,而金輪法王本人
也只練到第十層,但外功掌力強悍凶猛,聲稱有十龍十象的巨力,連出家後法名慈恩的鐵掌水上飄裘千仞亦不敵敗
亡\footnote{出自神雕俠侶第三十四回 排難解紛}。

\section{介紹}

龍象般若功載於龍象般若經上,份屬密宗裡至高無上的護法神功,共分十三層,第一層功夫十分淺易,縱是下愚之人,只要得到傳授,一二年中即能練成。第二層比第一層加深一倍,需用三四年時間。第三層又比第二層加深一倍時間。如此成倍遞增,越是往後精進,越是困難。待到第五層以後,每要再練深一層,往往需要耗費三十年以上的苦功。

由於龍象般若功修練不易,因而儘管密宗一門,高僧奇士歷代輩出,但卻沒有人練到超過第十層,北宋年間曾有一名藏邊高僧練至第九層,繼續勇猛精進第十層,卻因心魔驟起,無法自制,狂舞七日七夜,自終絕脈而死。直到金輪法王為雪敗給楊過、小龍女之恥時,回到蒙古後竟爾沖破第九層難關,達到第十層的境界\footnote{出自神雕俠侶第三十七回 三世恩怨}。後來在攻打襄陽城時,金輪法王也在火燒郭襄的高塔上憑藉此功力鬥楊過的黯然銷魂掌,終於不敵被打落高台,遭周伯通壓在火柱之下斃命\footnote{神雕俠侶第三十九回 大戰襄陽}。

在世紀新修版的《神雕俠侶》中,基於郭襄確實拜了金輪法王為師,龍象般若功似乎也由金輪法王傳授給郭襄,而
在高台之戰裡,金輪法王最在郭襄遇險時也捨命使用龍象般若功擊開火柱,力竭身亡。

%%% Local Variables: 
%%% mode: latex
%%% TeX-master: "../sample"
%%% End: 
