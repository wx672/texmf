\chapter{火焰刀}
火焰刀\footnote{参见维基百科 - \href{http://zh.wikipedia.org/wiki/\%E7\%81\%AB\%E7\%84\%B0\%E5\%88\%80}{火焰刀}}為
金庸武俠小說《天龍八部》中之武功,乃密教寧瑪派的神奇武功,可將虛無縹緲的掌氣凝刀刃之利,不可捉摸,能
殺人于無形\footnote{出自天龍八部第十章 劍气碧煙橫}。

\section{介紹}

火焰刀是鳩摩智在吐蕃密教寧瑪派出家,因與吐蕃國黑教邪徒鬥爭劇烈,從寧瑪派上師處學得之武技,能以內力凝聚於手掌掌緣,運氣送出劈砍敵手,所以雖屬掌法卻以刀為名。鳩摩智曾憑藉這門武功和天龍寺的六脈神劍陣劇鬥一場,後來也曾多次用於應敵。

在金庸進行修改後的新三版中,鳩摩智能從慕容博手裡獲得少林七十二絕技,就是用火焰刀的修練法訣交換,也被慕容博評說火焰刀動念即發,猶勝發勁緩慢的一陽指,並提及和火焰刀有相同妙用但可發發六種內力的六脈神劍,引起鳩摩智日後往天龍寺奪經的遠因。

%%% Local Variables: 
%%% mode: latex
%%% TeX-master: "../sample"
%%% End: 
