\chapter{獨孤九劍}
独孤九剑
\footnote{参见维基百科 - \href{http://zh.wikipedia.org/wiki/\%E7\%8D\%A8\%E5\%AD\%A4\%E4\%B9\%9D\%E5\%8A\%8D}{獨孤
    九劍}}是出現於金庸小说《笑傲江湖》上之劍法,令狐沖於思過崖遭遇到強敵田伯光,風清揚在後不願動手,
於是又一夜時間傳授令狐沖獨孤九劍心法\footnote{《笑傲江湖》第十回}。

\section{簡介}
劍法分做九大部分:總訣式、破劍式、破刀式、破槍式、破鞭式、破索式、破掌式、破箭式、破氣式,分別是依據不同兵器而生的對招方式,而就其本質來說,則可理解為「與九種不同兵器對陣時,所採用的攻防觀念」。其中最需要注意的,是使用掌法或其它拳腳功夫的對手,原因是這一類的對手不用兵刃,自然在拳腳與內力上有高超之處,而且武學修為也已到一境界,有無兵器已相差不多。

獨孤九劍無招,完全視對方招式而定,所以遇強則強。在笑傲江湖裡面,令狐沖曾多次被稱讚劍法精妙,包括東方不敗,原因是攻擊者自身劍法高深。

獨孤九劍意境乃跟隨中國哲學莊子,以無用之用乃為大用為原則,並非亂砍(風清揚強調過此點),而是仔細觀察對方招式,迅速找到破綻,攻其所必救,而攻擊之法沒有一定,完全視獨孤九劍之使用者的意向而定,而令狐沖在書裡使用的,多半是兩敗俱傷的打法,甚或以出拳的方式化解危機。

令狐沖靠獨孤九劍脫離險境,或精采對戰的橋段共有:思過崖-田伯光、五岳劍派大會-劍宗師叔封不平和成不憂、
破廟-十五人刺客、梅莊四友、任我行、岳不群 (已練成七十二路辟邪劍法) 、日月神教總壇-東方不敗等等。

\section{九劍概說}

\paragraph{總訣式}

總訣,是三千餘字的入門口訣,亦為其後八式的變化總要。其部份內容為:「歸妹趨無妄,無妄趨同人,同人趨大有。甲轉丙,丙轉庚,庚轉癸。子丑之交,辰巳之交,午未之交。風雷是一變,山澤是一變,水火是一變。乾坤相激,震兌相激,離巽相激。三增而成五,五增而成九……」

\paragraph{破劍式}

\begin{itemize}
\item 目的:破解天下各門各派之劍法。
\end{itemize}

\paragraph{破刀式}

\begin{itemize}
\item 目的:破解單刀、雙刀、柳葉刀、鬼頭刀、大砍刀、斬馬刀等種種刀法。
\item 要旨:以輕御重,以快制慢。
\end{itemize}

\paragraph{破槍式}

\begin{itemize}
\item 目的:破解長槍、大戟、蛇矛、齊眉棍、狼牙棒、白蠟杆、禪杖、方便鏟等長兵器。
\end{itemize}

\paragraph{破鞭式}

\begin{itemize}
\item 目的:鋼鞭、鐵鞭、點穴橛、拐子、蛾眉刺、匕首、板斧、鐵牌、八角槌、鐵錐等短兵器。
\end{itemize}

\paragraph{破索式}

\begin{itemize}
\item 目的:破解長索、軟鞭、三節棍、鏈子槍、鐵鏈、魚網、飛錘等軟兵器。
\end{itemize}

\paragraph{破掌式}

\begin{itemize}
\item 目的:破的是拳腳指掌上的功夫,對方既敢以空手來鬥自己利劍, 武功上自有極高造詣,手中有無兵器,
  相差已是極微。天下的拳法、腿法、指法、掌法繁複無比,這一劍“破掌式”,將長拳短打、擒拿點穴、鷹爪虎爪、
  鐵沙神掌,諸般拳腳功夫盡數包括在內。 
\end{itemize}

\paragraph{破箭式}

\begin{itemize}
\item 目的:則總羅諸般暗器,練這一劍時,須得先學聽 風辨器之術,不但要能以一柄長劍擊開敵人發射來的種種暗器,還須借力反打,以敵人射 來的暗器反射傷敵
\end{itemize}

\paragraph{破氣式}

\begin{itemize}
\item 目的:對付身具上乘內功之敵手。
\end{itemize}

\section{小說橋段}
令狐沖也曾問過風清揚,若兩個劍法無招的人對戰,何者為勝,風清揚也說不知道。令狐沖只有三次在使獨孤九劍時心驚膽跳:
\begin{itemize}
\item 第一次乃遭遇武當派的太極劍法:太極劍法乃圓融循環,以劍光藏住中心破綻,最後令狐沖憑推測冒險往中心一刺,令武當派掌門沖虛道長不得不退,僥倖獲勝,但因此在武功上大有進益,明白了:「敵人招數中之最強處,竟然便是最弱處,最強處都能擊破,其餘自是迎刃而解了」。
\item 第二次是與東方不敗的對戰:後者使用葵花寶典中武功,出招極快,破綻一閃即逝,令狐沖縱能視出東方不敗破綻亦不及攻擊,與任我行等人陷入苦戰。
\item 第三次是在華山山洞中,因眼不見物,無法找到對手破綻,最後意外靠魔教十長老的腿骨中的磷光得以見到對手武功,才得以施展劍法。
\end{itemize}

%%% Local Variables: 
%%% mode: latex
%%% TeX-master: "../sample"
%%% End: 
