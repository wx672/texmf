\chapter{吸星大法}
吸星大法\footnote{参见维基百科 -
  \href{http://zh.wikipedia.org/wiki/\%E5\%90\%B8\%E6\%98\%9F\%E5\%A4\%A7\%E6\%B3\%95}{吸星大法}}是
金庸小說《笑傲江湖》中的一個虛構內功。這套內功是將別人的內力吸收,將這些別人的內力化成自己體內的內力,
或將真氣排出。這吸星大法,創自北宋年間的逍遙派,分為北冥神功與化功大法兩路。後來從大理段氏及星宿派分
別傳落,合而為一,却又偏向化功大法,稱為吸星大法。 

\section{修練難處}

\begin{itemize}
\item 第一,是要散去全身內力,使得丹田中一無所有,只要散得不盡,或行錯了穴道,立時便會走火入魔,輕則全身癱瘓,從此成了廢人,重則經脈逆轉,七孔流血而亡\footnote{取自《笑傲江湖》第二十二章【脫困】}。
\item 第二,散功之後,又須吸取旁人的真氣,貯入自己丹田,再依法驅入奇經八脈以供己用。這一步本來也十分艱難,自己內力已然散盡,再要吸取旁人真氣,豈不是以卵擊石,徒然送命?
\end{itemize}

\section{修練缺陷}

這種內功之中有幾個重大缺陷,初時修練時不會覺,其後禍患卻慢慢顯露出來。如果修練後不理會它,終有一日會
得毒火焚身。那些吸取而來的他人功力,會突然反噬,吸來的功力愈多,反撲之力愈大\footnote{取自《笑傲江湖》
  第二十二章【脫困】}。 

\section{解決方法}

\begin{itemize}
\item 第一,是任我行在梅莊地牢中被困的十二年內發現。書中只提及任我行懂得化解,任我行叫其為融功,但並未提及使用方法。但基本應為使用本身內勁強行壓制。此法雖看似能化解功力反噬之虞,但終究不能長久。
\item 第二,是少林寺方證大師傳授給令狐沖的《易筋經》。在《笑傲江湖》第四十章【曲諧】裡說到,風清揚命方證代傳口訣-「華山內功心法」,但在最後任盈盈對令狐沖說:「沖哥,你到今日還是不明白,你所學的,便是少林派的《易筋經》內功。」
\end{itemize}

%%% Local Variables: 
%%% mode: latex
%%% TeX-master: "../sample"
%%% End: 
