\chapter{一陽指}
一陽指\footnote{参见维基百科 - \href{http://zh.wikipedia.org/wiki/\%E4\%B8\%80\%E9\%99\%BD\%E6\%8C\%87}{《一阳指》}}是金庸武俠小說《射鵰英雄傳》、《神鵰俠侶》與《天龍八部》的一套武功,為雲南大理段氏的獨家武學,擊中時的威力十分巨大。在《射鵰英雄傳》中,「天下五絕」之一的南帝段智興,便是以此武功稱霸天下\cite{yiyangzhi}。

\section{概述}

一陽指為大理段氏的家傳武學,段氏皇族段正明、段正淳、段延慶、段智興皆身具一陽指武功。《射鵰英雄傳》中,曾提及一燈大師段智興已修練到「登峰造極、爐火純青」的境界,並且以一陽指與「天下五絕」之首王重陽交換先天功。一陽指與先天功也是「西毒」歐陽鋒獨門絕學蛤蟆功的剋星。

一陽指有指法以及其獨門的內力,一燈大師曾以多年深厚的一陽指內力替身受重傷的黃蓉治療,但所耗損的內力亦需花多年時間回復,最後一燈幸得《九陰真經》的武功在短時間內恢復。一燈大師將一陽指武功傳給門下弟子「漁樵耕讀」,其中「書生」朱子柳甚至將一陽指與中國書法融合,名為「一陽書指」。《神鵰俠侶》中,「農夫」武三通將一陽指武功分別傳授給兒子武敦儒與武修文,但造詣不深。

《天龍八部》中,大理皇帝段正明與皇太弟段正淳也會使一陽指,「四大惡人」之首段延慶的一陽指造詣亦不遜於
段正明。大理天龍寺中,枯榮大師與門下弟子皆有修練一陽指,其中曾提及鎮寺之寶六脈神劍,便是以一陽指渾厚
的內力化成無形劍氣攻敵,而一陽指在最高第一品時,其境界亦不遜於其他指法武功。

%%% Local Variables: 
%%% mode: latex
%%% TeX-master: "../sample"
%%% End: 
