\chapter{天山六陽掌}
天山六陽掌
\footnote{参见维基百科 - \href{http://zh.wikipedia.org/wiki/\%E5\%A4\%A9\%E5\%B1\%B1\%E5\%85\%AD\%E9\%99\%BD\%E6\%8E\%8C}{
    维基百科-天山六陽掌}}為金庸武俠小說《天龍八部》中之武功,與天山折梅手皆屬逍遙派中靈鷲宮一脈的武學,
共分六式,每招名稱中均帶一「陽」字,故名「六陽掌」也是生死符的破解法門\footnote{出自天龍八部第三十七章 同一笑 到頭万事俱空}。

\section{簡介}
天山六陽掌乃逍遙派不傳之秘,在逍遙子三名徒兒中除長徒天山童姥習得外,據李秋水的話意,無涯子似乎也練有這門掌功。

在天山童姥被虛竹救往西夏皇宮冰庫藏匿期間,當天山折梅手傳授給虛竹後,為對付李秋水便打算一併把天山六陽掌教給他,不料虛竹不肯助其殺人,堅不肯學,甚至起意離開。

為此,天山童姥用謀在虛竹身上種落生死符,再假稱教他破解之道,把生死符剋星天山六陽掌的功夫傾囊相授\footnote{出自天龍八部第三十六章 夢里真 真語真幻}。因而在李秋水追入冰庫要殺天山童姥時,虛竹無意間就使出了第二招「陽春白雪」、第七招「陽關三疊」以及「陽歌鉤天」三招接下李秋水的攻擊。

在天山童姥死後,虛竹繼承靈鷲宮,因可憐三十六洞、七十二島眾人身受生死符苦楚,遂使用天山六陽掌幫他們拔除生死符,贏得眾人真心相隨,後來也憑藉天山六陽掌力鬥丁春秋,用生死符逼他就範受擒。

%%% Local Variables: 
%%% mode: latex
%%% TeX-master: "../sample"
%%% End: 
