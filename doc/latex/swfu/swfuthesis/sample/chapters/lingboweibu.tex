\chapter{凌波微步}
凌波微步
\footnote{参见维基百科 - \href{http://zh.wikipedia.org/wiki/\%E5\%87\%8C\%E6\%B3\%A2\%E5\%BE\%AE\%E6\%AD\%A5}{凌波
    微步}}是金庸在其作品《天龙八部》中虚构的一种轻功。

\section{簡介}
凌波微步乃是一門極上乘的武功,每一步踏出,全身行動與內力息息相關,決非單是邁步行走而。它名出于曹植《洛神赋》—“休迅飞凫,飘忽若神。陵波微步,罗袜生尘。动无常则,若危若安。进止难期,若往若还。转眄流精,光润玉颜。含辞未吐,气若幽兰。华容婀娜,令我忘餐。”原意是形容洛神体态轻盈,浮动于水波之上,缓缓行走。其中“休迅飞凫,飘忽若神”及“动无常则,若危若安。进止难期,若往若还”可作为这种武功的注解。

段譽在大理國無量山琅嬛福地給玉像磕頭後,得到逍遙派武功精要秘籍北冥神功,凌波微步位于卷轴之末,要待人練成「北冥神功」,吸人內力,自身內力已頗為深厚之後再練這一頂級武功。
\section{步法}
凌波微步是依照周易六十四卦的方位而演變的武功步法,步法甚怪,須得憑空轉一個身或躍前縱後、左竄右閃,方
合於捲上的步法。禦敵對陣時只需按六十四卦步法行走而無需顧忌對手的存在,是一種我行我素、天馬行空的上乘
功夫。另外凌波微步是以動功修習內功,腳步踏遍六十四卦一個周天,內息自然而然的也轉了下個周天,因此每走
一遍,內力便有一分進益。“凌波微步”、“北冥神功”和“六脈神劍”也成為段譽三大護身武功。

\section{方位}

明夷→賁→既濟→家人……\par
→中孚→既濟→泰→蠱……\par
→井→訟→蠱……\par
→豫→觀……\par
→無妄

%%% Local Variables: 
%%% mode: latex
%%% TeX-master: "../sample"
%%% End: 
