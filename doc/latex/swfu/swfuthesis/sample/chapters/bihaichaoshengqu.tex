\chapter{碧海潮生曲}
《碧海潮生曲》\footnote{参见维基百科 - \href{http://zh.wikipedia.org/wiki/\%E7\%A2\%A7\%E6\%B5\%B7\%E6\%BD\%AE\%E7\%94\%9F\%E6\%9B\%B2}{《碧海潮生曲》}}最初是金庸武俠小說中黃藥師所創的武功乐曲。

東邪黃藥師精通琴、棋、書、畫、醫、卜、兵、陣,他自創的這首《碧海潮生曲》,表面上簫聲聽似模仿大海潮浪之聲,其實內藏極高度致命武功,聲情致飄忽,纏綿宛轉,若在無防備之下聆聽則難以自制,不住手舞足蹈,甚至胡亂抓搔頭臉。


台灣作曲家劉學軒以《碧海潮生曲》為主題,使用曲笛與古箏首度將此乐曲搬上舞台。首演時間為2006年9月16日,地
點為台灣國家演奏廳,於國家國樂團「笑傲江湖」音樂會中世界首演。

%%% Local Variables: 
%%% mode: latex
%%% TeX-master: "../sample"
%%% End: 
