\documentclass{swfuthesis}

\addbibresource{thesis.bib}    % 参照教程自己去写一个.bib文件

\title{论文标题}
\enTitle{英文标题}
\author{作者姓名}
\enAuthor{英文姓名}
\Advisor{指导教师姓名}
\AdvisorTitle{指导教师职称}
\Month{六}
\Year{二〇一九}
\Subject{计算机科学与技术专业}    %专业名称(比如 电子信息工程专业)

\begin{document}

\maketitle

\begin{abstract} % 摘要
  本文是关于……
\end{abstract}

\begin{keyword} % 关键词
  关键词,关键词,关键词……
\end{keyword}

\begin{EAbstract} % 英文摘要
  This project is about...
\end{EAbstract}

\begin{EKeyword} % 英文关键词
  keyword, keyword, keyword...
\end{EKeyword}

\tableofcontents     % 目录
\listoffigures       % 插图目录,可以没有
\listoftables        % 表格目录,可以没有
\cleardoublepage % keep this line
\pagenumbering{arabic}

% 参考教程,在chapters目录中单独写各章
\include{chapters/ch1-intro}
\include{chapters/ch2-design}
\include{chapters/ch3-implementation}
\include{chapters/ch4-test}

%%%%% appendix (参考文献、指导教师简介、鸣谢、附录)
\appendix % keep this line
\makebib % 参考文献

\begin{advisorInfo} % 指导教师简介
  王晓林,男,50岁,硕士,讲师,毕业于英国格林尼治大学,分布式计算系统专业。现任
  西南林业大学计信学院教师。执教Linux、操作系统、网络技术等方面的课程,有丰富的Linux教学和系统管理经验。
\end{advisorInfo}

\begin{acknowledgment} % 致谢
  我要感谢党,感谢政府老能让我情绪稳定……
\end{acknowledgment}

%%%%% 附录章节
\singlespacing
\include{chapters/ch5-app-code}


\end{document} % 结束。不要动下面几行!

%%% Local Variables:
%%% mode: latex
%%% TeX-master: t
%%% End:
