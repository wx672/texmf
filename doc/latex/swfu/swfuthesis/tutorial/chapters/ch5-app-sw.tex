\chapter{与排版论文相关的软件清单}%
\footnotetext{这份清单是用如下命令获得的:\par
  \texttt{aptitude search '\~{}i "texlive|fonts|latexmk|pygments|emacs|fcitx|xterm|sawfish"' > pkg-list}}
\label{app:pkg}

\begin{enumerate}
\item emacs25 - GNU Emacs editor (with GTK+ GUI support)
\item emacs25-common-non-dfsg - GNU Emacs common non-DFSG items, including the core documentation
\item fcitx - Flexible Input Method Framework
\item fcitx-pinyin - Flexible Input Method Framework - classic Pinyin engine
\item fonts-noto - metapackage to pull in all Noto fonts
\item fonts-noto-cjk-extra - "No Tofu" font families with large Unicode coverage (CJK all weight)
\item fonts-arphic-ukai - "AR PL UKai" Chinese Unicode TrueType font collection Kaiti style
\item fonts-symbola - symbolic font providing emoji characters from Unicode 9.0
\item fonts-tlwg-purisa - Thai Purisa font (dependency package)
\item latexmk - Perl script for running LaTeX the correct number of times
\item sawfish - window manager for X11
\item sawfish-merlin-ugliness - More flexible functions for sawfish
\item texlive-bibtex-extra - TeX Live: BibTeX additional styles
\item texlive-extra-utils - TeX Live: TeX auxiliary programs
\item texlive-generic-extra - TeX Live: transitional dummy package
\item texlive-generic-recommended - TeX Live: transitional dummy package
\item texlive-lang-chinese - TeX Live: Chinese
\item texlive-lang-english - TeX Live: US and UK English
\item texlive-latex-recommended - TeX Live: LaTeX recommended packages
\item texlive-luatex - TeX Live: LuaTeX packages
\item texlive-xetex - TeX Live: XeTeX and packages
\item python-pygments - syntax highlighting package written in Python   
\item xterm - X terminal emulator
\end{enumerate}

\vspace{2ex}

注意,这是一个不完全的软件包列表。但是,在安装这些软件包时,其它一些被依赖的软件包会被自动
安装上。另外,Emacs的很多插件,比如\auctex{}, pdf-tools等,可以通过Emacs的插件管理模块来安装、
更新,在此没有列出。

%%% Local Variables:
%%% mode: latex
%%% TeX-master: "../tutorial"
%%% End:
