\chapter{完美的毕业论文}
\label{cha:thesis}

写毕业论文,像结婚一样,是一生一次的大事情。潦草而失败的论文,就像失败的婚姻,总要一次次地
返工,在痛苦中煎熬,直到你遇见ta,带你摆脱迷茫、痛苦,登上幸福的彼岸\footnote{我说的当然
  是\LaTeX{}。}。

\begin{itemize}
\item[] 写普通文章要用 \ltx{\documentclass{article}};
\item[] 写报告要用 \ltx{\documentclass{report}};
\item[] 写书要用 \ltx{\documentclass{book}};
\item[] 写信要用 \ltx{\documentclass{letter}};
\item[] 那么写毕业论文自然要用毕业论文模版了:\ltx{\documentclass{swfuthesis}}。
\end{itemize}

如此关系人生的大事情,当然要为它专门建立一个目录吧。目录建好了,把论文模板 \texttt{swfuthesis.cls} 文件拷贝
进去。然后就可以用它来写你的论文了。

\section{Class文件}
\label{sec:class}

Class文件\footnote{也就是后缀为\texttt{.cls}的文件,也就是我们常说的模板文件。},它决定了你
的文章样式,比如说,纸张尺寸、页边距、行距、字距、字体、标题样式等等在class文件中都做了设
置。除此之外,我们在写\texttt{tex}文件的过程中用到的命令(Macro)也都是class文件提供的。

这里有一个值得注意的概念,排版这件事情,是由排版软件根据你(在文章中输入)的命令来进行的。
只要你的命令正确,比如不要把\ltx{\author}误写成\ltx{\auther},文章的格式就必然是正确的。这
就是排版软件和字处理软件(比如MS-Word)的区别所在。利用字处理软件来写文章,你不得不既操心文
章的内容,也操心文章的格式。而利用排版软件,比如\LaTeX{}来写文章,你只需要关心文章的内容,
而格式的事情,排版软件会根据你的命令来替你完成。所以,你输入的命令必须正确、合法、合情理才
行。

排版软件只能理解class文件中提供的命令\footnote{我说谎了,实际上,在\texttt{tex}文件中,你可
  以利
  用 \ltx{\newcommand{}}和 \ltx{\renewcommand{}}来
  定义自己的命令。但对于初学者来说,你暂时还不必操心这个,class文件所提供的命令应该足以应付
  你目前的需求了。},所以,我们当然要对这些命令有个基本的了解。简而言
之,\texttt{swfuthesis.cls}只在普通\LaTeX{}格式的基础上额外提供了两个命令:一个
是\ltx{\swfusetup},用法如下,应该不用解释了吧。

\begin{latexcode}
\swfusetup{
  Title        ={怎样用LaTeX排版一个双面打印,标题长得没必要的毕业论文}, % 论文标题。
  Author       ={王晓林}, % 作者姓名
  Signature    ={\includegraphics[width=6em]{signature.pdf}}, % 作者签名(用于原创声明页)
  enTitle      ={How to write your thesis in \LaTeX}, % 论文标题(英文)
  enAuthor     ={WANG Xiaolin}, % 作者姓名(英文)
  ID           ={20161152888}, % 学号
  Year         ={2022},
  Month        ={5},
  Date         ={20},
  Major        ={数据科学与大数据技术},% 专业
  Advisor      ={指导教师甲(职称),指导教师乙(职称)},%
  Reviewer ={Name (Title)}, % 评阅人
}
\end{latexcode}

另一个命令是\ltx{\makebib},用于生成参考文献页。把它放
到\ltx{\appendix}后面就行了。除了上述两个命令,\texttt{swfuthesis.cls}还
提供了如下一些Environments:

\begin{multicols}{2}
  \begin{enumerate}
  \item \verb'abstract':中文摘要
  \item \verb'keyword':中文关键词
  \item \verb'EAbstract':英文摘要
  \item \verb'EKeyword':英文关键词
  \item \verb'advisorInfo':指导教师简介
  \item \verb'acknowledgment':致谢
  \item \verb'xxxcode':插入代码。比如,你想插入Python代码,就
    把\texttt{xxx}换成\texttt{python};如果是C代码就换成\texttt{c}。此外还支
    持CPP、Shell、Latex、Bibtex、Gas、Nasm,如果这些还不够,你可以照猫画虎地
    在\texttt{swfuthesis.cls}里增加你需要的编程语言。
  \end{enumerate}
\end{multicols}

\section{模板的使用}

具体如何使用这些命令呢?简而言之,\Ctrl{x} \Ctrl{f},输入一个崭新的文件名,\Ctrl{j}。现在,面对空无一字
的Emacs窗口,你有生以来最重要的一篇文章就要开篇了,你在思考什么?其实什么都不用想,直接
敲 \texttt{thesis} \LKeyTab{}(Tab),一个论文框架就呈现在你的眼前了,你要做的就是用你前面学到的
那些\LaTeX{}命令来填空。

在本文开始的时候(第\pageref{p:yasnippet}页),我们提到过“{\LKeyTab}大法”是\texttt{yasnippet}的
贡献。所以,你在期待\LKeyTab{}键能带来奇迹之前,必须先确保你的系统里已经装好了
\texttt{yasnippet}。如果还没装上,那么就执行下面的命令\cite{aptitude}:

\begin{codeblock}
  \begin{shellcode}
sudo apt update && sudo apt upgrade
sudo apt install yasnippet yasnippet-snippets
  \end{shellcode}
\end{codeblock}

装好之后,重新加载Emacs的配置文件。很简单,在Emacs窗口里,敲

\begin{itemize}
\item[] \Meta{x} \texttt{load-file} \Ctrl{j} \verb'~/.emacs' \Ctrl{j}
\end{itemize}

跟着本教程写论文的前提条件就是,Emacs必须工作正常,如果发现问题必须及时处理,不能将就。尤其
是在入门阶段,不要指望用一个别扭的Emacs来顺畅地写出完美的论文。

假设你的Emacs环境良好,按{\LKeyTab}键之后,本来空荡荡的Emacs buffer里现在就有了一个论文框架,而且
光标就停在你要填空的第一个位置,也就是“\ltx{\title{论文标题}}”的花括号里。这时,你只要键入
任何文字(当然应该是你自己的论文标题),(感谢\texttt{yasnippet})花括号里的“论文标题”四字
就会被自动替换掉。写好了论文标题,还是按{\LKeyTab}键,光标会自动跳到下一个你需要填空的位置。如
此一直{\LKeyTab}下去,直到光标不再跳开了,那就是你该写论文第一章内容的时候了。

论文框架里有足够多的注释,再加上你在前几章学到的本领,我相信写出一个规范、漂亮的毕业论文应该是手
到擒来的事情了。

\section{生成参考文献}

在第\ref{sec:ref}节中我们已经介绍过如何处理参考文献。大致步骤:
\begin{enumerate}
\item 准备好一个\texttt{.bib}文件(假设名字叫\texttt{myref.bib});
\item 在\texttt{.tex}文件中做三件事:
  \begin{enumerate}
  \item 在\texttt{preamble}中加一句 \ltx{\addbibresource{myref.bib}};
  \item 在正文中适当的地方利用\ltx{\cite{}}命令引用\texttt{myref.bib}文件中的参考文献条目;
  \item 在\ltx{\appendix}后面加上\ltx{\makebib}。
  \end{enumerate}
\item 用\texttt{latexmk}来编译\texttt{.tex}文件。
\end{enumerate}

\texttt{tutorial.bib}就是本文所用到的\texttt{.bib}文件,
你可以对照着第\pageref{p:ref}页的参考文献部分看看,找找感觉。

\paragraph*{这个文件是怎么来的?}

当然可以手写。\texttt{bib}文件的格式清晰易懂,照猫画虎地手写并不困难。但生活可以更轻松,如
果你引用的书籍、资料不是太冷门,那么通常只要google一下「书名 bibtex」,就可以找到相应
的\texttt{bib}条目了。有时你甚至可以从网上下载到现成的bib文件,比如全
套RFC\footnote{Request for
  Comments. \url{https://en.wikipedia.org/wiki/Request_for_Comments}}的\texttt{bib}文件。

\paragraph*{怎样使用这个文件?}

很简单,三件事:

\begin{enumerate}
\item 第一件事不用做,因为在我们的论文模板文件(\texttt{swfuthesis.cls})里已经做了。在模板
  文件里有如下一行:
\begin{latexcode}
\RequirePackage[backend=biber,style=gb7714-2015]{biblatex}
\end{latexcode}
  也就是说,帮我们排版参考文献的是\texttt{biblatex}宏包;
\item 在\texttt{tex}文件(你的论文)的preamble部分,也就是在\ltx{\document}之前,加上如下一
  行:
\begin{latexcode}
\addbibresource{tutorial.bib}
\end{latexcode}
  很显然,这是在告诉\texttt{biblatex}从当前目录下的\texttt{tutorial.bib}文件里去读取参考文
  献信息。
\item 这件事也不用做了,因为在论文的框架文件(\texttt{tex}文件)里已经做好了。
  在\texttt{tex}文件的\texttt{appendix}(附录)部分,你能看到如下一行:
\begin{latexcode}
\makebib
\end{latexcode}
  这就是要求排版输出参考文献。
\end{enumerate}

\section{小结}

在此我们简单回顾一下,要写出漂亮的毕业论文所应具备的必要条件:

\begin{enumerate}
\item 一套高效的工作环境
  \begin{enumerate}
  \item Debian sid;
  \item \TeX{}Live;
  \item Emacs+\auctex{}+Yasnippet;
  \end{enumerate}
\item 对Emacs基础快捷键的熟练使用;
\item \LaTeX{}的基础知识;
\end{enumerate}

只要具备了以上条件,轻松地写出一份漂亮、规范的毕业论文应该不成问题了。
先到这里吧。以后如果有时间,我会继续介绍一下

\begin{enumerate}
\item 如何用\LaTeX{} Beamer做出漂亮的演示幻灯片;
\item 如何用Emacs org-mode快速生成PDF文件。
\end{enumerate}

另外,配合本教程,我制作了一个简单的小视频,\texttt{tutorial.mkv},它应该和本文放在同一个
目录下\footnote{\url{http://cs6.swfu.edu.cn/~wx672/swfcthesis/tutorial/}}。

关于本文的任何疑问和建议,欢迎反馈到\texttt{wx672ster@gmail.com}。

\begin{flushright}
  {\Huge \fontspec{Purisa}Happy \TeX{}ing!}
\end{flushright}

%%% Local Variables:
%%% mode: latex
%%% TeX-master: "../tutorial"
%%% End:
