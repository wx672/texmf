\begin{advisorInfo} % 指导教师简介
  三闾大学校长高松年是位老科学家。这“老”字的位置非常为难,可以形容科学,
  也可以形容科学家。不幸的是,科学家跟科学不大相同;科学家像酒,愈老愈
  可贵,而科学像女人,老了便不值钱。将来国语文法发展完备,终有一天可以
  明白地分开“老的科学家”和“老科学的家”,或者说“科学老家”和“老科学家”。
  现在还早得很呢,不妨笼统称呼。高校长肥而结实的脸像没发酵的黄面粉馒
  头,“馋嘴的时间”(edaxvetustas)咬也咬不动他,一条牙齿印或皱纹都没有。
  假使一个犯校规的女学生长得很漂亮,高校长只要她向自己求情认错,也许会
  不尽本于教育精神地从宽处分。这证明这位科学家还不老。他是二十年前在外
  国研究昆虫学的;想来三十年前的昆虫都进化成为大学师生了,所以请他来表
  率多士。他在大学校长里,还是前途无量的人。大学校长分文科出身和理科出
  身两类。文科出身的人轻易做不到这位子的。做到了也不以为荣,准是干政治
  碰壁下野,仕而不优则学,借诗书之泽,弦诵之声来休养身心。理科出身的人
  呢,就完全不同了。中国是世界上最提倡科学的国家,没有旁的国度肯这样给
  科学家大官做的。外国科学进步,中国科学家进爵。在国外,研究人情的学问
  始终跟研究物理的学问分歧;而在中国,只要你知道水电,土木,机械,动植
  物等等,你就可以行政治人——这是“自然齐一律”最大的胜利。理科出身的人当
  个把校长,不过是政治生涯的开始;从前大学之道在治国平天下,现在治国平
  天下在大学之道,并且是条坦道大道。对于第一类,大学是张休息的靠椅;对
  于第二类,它是个培养的摇篮 --- 只要他小心别摇摆得睡熟了。

  高松年发奋办公,夙夜匪懈,精明得真是睡觉还睁着眼睛,戴着眼镜,做梦都
  不含糊的。摇篮也挑选得很好,在平成县乡下一个本地财主家的花园里,面溪
  背山。这乡镇绝非战略上必争之地,日本人唯一豪不吝惜的东西——炸弹——也不
  会浪费在这地方。所以,离开学校不到半里的镇上,一天繁荣似一天,照相铺,
  饭店,浴室,戏院,警察局,中小学校,一应俱全。今年春天,高松年奉命筹
  备学校,重庆几个老朋友为他饯行,席上说起国内大学多而教授少,新办尚未
  成名的学校,地方偏僻,怕请不到名教授。高松年笑道:“我的看法跟诸位不同。
  名教授当然好,可是因为他的名望,学校沾着他的光,他并不倚仗学校里地位。
  他有架子,有脾气,他不会全副精神为学校服务,更不会绝对服从当局指挥。
  万一他闹别扭,你不容易找替人,学生又要借题目麻烦。我以为学校不但造就
  学生,并且应该造就教授。找到一批没有名望的人来,他们要借学校的光,他
  们要靠学校才有地位,而学校并非非有他们不可,这种人才真能跟学校合为一
  体,真肯为公家做事。学校也是个机关,机关当然需要科学管理,在健全的机
  关里,决没有特殊人物,只有安分受支配的一个个单位。所以,找教授并非难
  事”。大家听了,倾倒不已。高松年事先并没有这番意见,临时信口胡扯一阵。
  经朋友们这样一恭维,他渐渐相信这真是至理名言,也对自己倾倒不已。他从
  此动不动就发表这段议论,还加上个帽子道:“我是研究生物学的,学校也是个
  有机体,教职员之于学校,应当像细胞之于有机体” --- 这段至理名言更变而为科
  学定律了。\qquad{}------ 《围城》,钱钟书
\end{advisorInfo}

%%% Local Variables:
%%% mode: LaTeX
%%% TeX-master: "../tutorial"
%%% End:
