\documentclass{article}  % documentclass可以是
                           % article, book, report, letter...
\begin{document}         % 文章的开始
\title{How to Structure a \LaTeX{} Document}
\author{Andrew Roberts\\
  School of Computing,\\
  University of Leeds,\\
  Leeds,\\
  United Kingdom,\\
  LS2 1HE\\
  \emph{andyr@comp.leeds.ac.uk}}
\date{\today}
\maketitle

\begin{abstract} 
  In this article, I shall discuss some of the fundamental
  topics in producing a structured document.  This document
  itself does not go into much depth, but is instead the
  output of an example of how to implement structure. Its
  \LaTeX{} source, when in used with my tutorial provides
  all the relevant information.
\end{abstract}

\section{Introduction}
\label{sec:introduction}

This small document is designed to illustrate how easy it is to create a well structured
document within \LaTeX\cite{lamport94}.  You should quickly be able to see how the article
looks very professional, despite the content being far from academic.  Titles, section
headings, justified text, text formatting etc., is all there, and you would be surprised
when you see just how little markup was required to get this output.

\section{Structure}
\label{sec:structure}

One of the great advantages of \LaTeX{} is that all it needs to know is
the structure of a document, and then it will take care of the layout
and presentation itself.  So, here we shall begin looking at how exactly
you tell \LaTeX{} what it needs to know about your document.

\subsection{Top Matter}
\label{sec:top-matter}

The first thing you normally have is a title of the document, as well as
information about the author and date of publication.  In \LaTeX{} terms,
this is all generally referred to as \emph{top matter}.

\subsubsection{Article Information}
\label{sec:article-information}

\begin{itemize}
\item \verb|\title{}| --- The title of the article.
\item \verb|\date| --- The date. Use:
  \begin{itemize}
  \item \verb|\date{\today}| --- to get the date that the document is typeset.
  \item \verb|\date{}| --- for no date.
  \end{itemize}
\end{itemize}

\subsubsection{Author Information}
\label{sec:author-information}
  
The basic article class only provides the one command:
\begin{itemize}
\item \verb|\author{}| --- The author of the document.
\end{itemize}

It is common to not only include the author name, but to insert new lines (\verb|\\|)
after and add things such as address and email details.  For a slightly more logical
approach, use the AMS article class (\emph{amsart}) and you have the following extra
commands:
  
\begin{itemize}
\item \verb|address| --- The author's address.  Use the new line command (\verb|\\|) for
  line breaks.
\item \verb|thanks| --- Where you put any acknowledgments.
\item \verb|email| --- The author's email address.
\item \verb|urladdr| --- The URL for the author's web page.
\end{itemize}

\subsection{Sectioning Commands}
\label{sec:sectioning-commands}
  
The commands for inserting sections are fairly intuitive.  Of course, certain commands are
appropriate to different document classes. For example, a book has chapters but a article
doesn't.
  
% A simple table. The center environment is first set up,
% otherwise the table is left aligned.  The tabular environment
% is what tells Latex that the data within is data for the table.

\begin{center}
  \begin{tabular}{|l|l|}
    \hline 
    Command & Level \\ \hline
    \verb|\part{}| & -1 \\
    \verb|\chapter{}| & 0 \\
    \verb|\section{}| & 1 \\
    \verb|\subsection{}| & 2 \\
    \verb|\subsubsection{}| & 3 \\
    \verb|\paragraph{}| & 4 \\
    \verb|\subparagraph{}| & 5 \\
    \hline
  \end{tabular}
\end{center}

Numbering of the sections is performed automatically by \LaTeX{}, so don't bother adding
them explicitly, just insert the heading you want between the curly braces.  If you don't
want sections number, then add an asterisk (*) after the section command, but before the
first curly brace, e.g., \verb|section*{A Title| \verb|Without Numbers}|.

\begin{thebibliography}{99}
\bibitem{lamport94}
  Leslie Lamport,
  \emph{\LaTeX: A Document Preparation System}.
  Addison Wesley, Massachusetts,
  2nd Edition,
  1994.
  \bibitem{simple}
    Andrew Roberts,
    \emph{A simple article to illustrate document structure}.
    Wikibooks,
    2003.
\end{thebibliography} %Must end the environment
\end{document}           % 文章的结束

%%% Local Variables:
%%% mode: latex
%%% TeX-master: t
%%% End:
