\documentclass{swfulabreport}
\deflength{\fboxsep}{1pt}
\usepackage{zhlipsum}

\swfusetup{%
%  Title        ={《计算机网络》课程实习}, % 课程名称
  Title        ={综合实习二}, % 课程名称
  Author       ={张三}, % 
  ID           ={20231159xxx}, % 学号
  Class     = {计科2023级专升本2班},%
  Tutor      ={王晓林}, %指导教师
  Year         ={\the\year},%
  Month        ={6},%
  Date         ={30},%
  % Major        ={计算机科学与技术2023级专升本2班},%专业班级
  Reviewer  = {\includegraphics[width=12mm]{wangxiaolin}},%
  ReviewDate={2024/7/8},%
  Mark      =,%
  LabDate   = {2024/6/28 - 2024/6/30},%
  LabDays   = {3},%
  Lab       = {经管楼219},%
  Heads     = {74},%
  Grouping  = {不分组},%
  Why       = {%实验目的
    掌握Linux平台上常用网络工具的使用,并理解基本的网络编程过程。
    \begin{enumerate}
    \item 掌握用tcpdump捕获并分析数据包
    \item 掌握用netcat完成简单的网络会话
    \item 掌握基本的网络编程
    \end{enumerate}
  },%
  Req      = {%实验要求
    \begin{enumerate}
    \item 在Linux平台完成所有实验
    \item 在Linux平台完成实验报告
    \item 努力尝试用英文撰写实验报告
    \item 将实验作业及报告以tgz格式打包,并上传到指定教学网站
    \item 迟交报告将被扣分
    \end{enumerate}
  },
  What      = {%实验内容
    详见《实验指导书》。
  },%
  Sched       = {%课程实习安排
    \begin{description}
    \item[2024/6/28:] 学习使用tmux,ttyrec等实验所需的命令
      行工具。学习使用ip, tcpdump, netcat, curl, ss, nmap等网络工具。
    \item[2024/6/29:] 完成用tcpdump捕获、分析数据包,用netcat实现网络协
      议会话,以及网络编程等规定实验项目。
    \item[2024/6/30:] 完成所有实验,并撰写实验报告。
    \end{description}
  },%
  Sumup       = {%实习总结
    你自己写三、五百字。
  },%
}

\begin{document}

\maketitle % 封面
\clearpage
\section{Packet analysis}

Upon running the following command:

\begin{shcode}
sudo tcpdump -ilo -nnvvvxXKS -s0 port 3333
\end{shcode}

the following packet is captured:

\begin{outputcode}
08:34:10.790666 IP (tos 0x0, ttl 64, id 12824, offset 0, flags [DF],
proto TCP (6), length 64)
  127.0.0.1.46668 > 127.0.0.1.3333: Flags [P.], seq 2400005725:2400005737,
ack 373279396,
  win 512, options [nop,nop,TS val 3259949783 ecr 3259896343], length 12
      0x0000:  4500 0040 3218 4000 4006 0a9e 7f00 0001  E..@2.@.@.......
      0x0010:  7f00 0001 b64c 0d05 8f0d 2e5d 163f caa4  .....L.....].?..
      0x0020:  8018 0200 fe34 0000 0101 080a c24e e2d7  .....4.......N..
      0x0030:  c24e 1217 6865 6c6c 6f20 776f 726c 640a  .N..hello.world.
\end{outputcode}

\begin{enumerate}
\item Tell me the meaning of each option used in the previous command.
  \begin{description}
  \item[-i:] Specify the network interface card to listen from.
  \item[-nn:] Don't print addresses.
  \item[-vvv:] Verbose output.
  \item[-x:] When parsing and printing, in addition to printing the
    headers of each packet, print the data of each packet, including
    its link level header, in hex.
  \item[-X:] When parsing and printing, in addition to printing the
    headers of each packet, print the data of each packet (minus its
    link level header) in hex and ASCII.  This is very handy for
    analysing new protocols.  In the current implementation this flag
    may have the same effect as -XX if the packet is truncated.
  \item[-S:] Print absolute, rather than relative, TCP sequence
    numbers.
  \item[-K:] Don't attempt to verify IP, TCP, or UDP checksums.  This
    is useful for interfaces that perform some or all of those
    checksum calcula‐ tion in hardware; otherwise, all outgoing TCP
    checksums will be flagged as bad.
  \item[-s0:] Setting snaplen to 0 sets it to the default of 262144,
    for backwards compatibility with recent older versions of tcpdump.
  \end{description}
  
\item Please analyze this captured packet and explain it to me as
  detailed as you can.
  \begin{description}
  \item[Answer:] \fbox{4}means ipv4; \fbox{5}is the header length in
    word; The \fbox{00} following \fbox{45} are the ToS bits;
    \fbox{0040} is the total length of this IPv4 packet; \fbox{3218}
    is the identification field; \fbox{4000} are the 2 flag bits plus
    the fragment offset field; \fbox{4006} means TTL is 0x40, 0x06
    means TCP; \fbox{0a9e} is the header checksum; \fbox{7f00 0001} is
    the IPv4 address of the loopback interface; \fbox{b64c} is the
    source port number; \fbox{0d05} is the destination port number,
    3333 in decimal; 
  \end{description}
\end{enumerate}

\section{HTTP}

\begin{enumerate}
\item Write a simple script showing how HTTP works (you need
  \texttt{curl});

  \begin{shcode}
    #!/bin/bash

    echo 'Hello, world!'
  \end{shcode}

\item Record your HTTP demo session with \texttt{ttyrec}.
\end{enumerate}

\section{Socket programming}

\subsection{TCP}

\begin{ccode}
/* A simple TCP server written in C */

/* Your code */
\end{ccode}

\begin{ccode}
/* A simple TCP client written in C */

// Your code
\end{ccode}

\subsection{UDP}

\begin{ccode}
/* A simple UDP server written in C */

// Your code
\end{ccode}

\begin{ccode}
/* A simple UDP client written in C */

// Your code
\end{ccode}

\section{Questions}

List at least 5 problems you've met while doing this work. When
listing your problems, you have to tell me:
\begin{enumerate}
\item Description of this problem. For example,
  \begin{itemize}
  \item What were you trying to do before seeing this problem?
  \end{itemize}
\item How did you try solving this problem? For example,
  \begin{itemize}
  \item Did you google? web links?
  \item Did you read the man page?
  \item Did you ask others for hints?
  \end{itemize}
\end{enumerate}

\subsection{Problems}

\begin{enumerate}
\item

\item

\item

\item

\item
\end{enumerate}

\end{document}
%%% Local Variables:
%%% mode: latex
%%% TeX-master: t
%%% End:
