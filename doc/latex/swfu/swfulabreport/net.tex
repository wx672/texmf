\documentclass{swfulabreport}

\usepackage{zhlipsum}

\swfusetup{%
  Title        ={《计算机网络》课程实习}, % 课程名称
  Author       ={张三}, % 
  ID           ={20231152xxx}, % 学号
  Year         ={\the\year},%
  Month        ={\the\month},%
  Date         ={\the\day},%
  Major        ={计算机科学与技术2023级},%专业班级
  Advisor      ={王晓林}, %指导教师
  Reviewer  = {\includegraphics[width=12mm]{wangxiaolin}},%
  ReviewDate= \today,%
  Mark      = {100},%
  LabDate   = {\today},%
  LabDays   = {3},%
  Lab       = {经管楼219},%
  Class     = {计科2023级},%
  Mates     = {88},%
  Group     = {不分组},%
  Why       = {%实验目的
    掌握Linux平台上常用网络工具的使用,并理解基本的网络编程过程。
    \begin{enumerate}
    \item 掌握用tcpdump捕获并分析数据包;
    \item 掌握用netcat完成简单的网络会话;
    \item 掌握基本的网络编程。
    \end{enumerate}
  },%
  Must      = {%实验要求
    \begin{enumerate}
    \item 在Linux平台完成所有实验;
    \item 在Linux平台完成实验报告;
    \item 努力尝试用英文撰写实验报告;
    \item 将实验作业及报告以tgz格式打包,并上传到指定教学网站;
    \item 迟交报告将被扣分。
    \end{enumerate}
  },
  What      = {%实验内容
    详见《实验指导书》。
  },%
  How       = {%课程实习安排
    \begin{description}
    \item[2024/7/1:] 学习使用tmux,ttyrec等实验所需的命令
      行工具。学习使用ip, tcpdump, netcat, curl, ss, nmap等网络工具。
    \item[2024/7/2:] 完成用tcpdump捕获、分析数据包,用netcat实现网络协
      议会话,以及网络编程等规定实验项目。
    \item[2024/7/3:] 完成所有实验,并撰写实验报告。
    \end{description}
  },%
  Sum       = {%实习总结
    你自己写三、五百字。
  },%
}

\begin{document}

\maketitle % 封面
\clearpage
\section{Packet analysis}

Upon running the following command:

\begin{shcode}
sudo tcpdump -ilo -nnvvvxXKS -s0 port 3333
\end{shcode}

the following packet is captured:

\begin{outputcode}
08:34:10.790666 IP (tos 0x0, ttl 64, id 12824, offset 0, flags [DF],
proto TCP (6), length 64)
  127.0.0.1.46668 > 127.0.0.1.3333: Flags [P.], seq 2400005725:2400005737,
ack 373279396,
  win 512, options [nop,nop,TS val 3259949783 ecr 3259896343], length 12
      0x0000:  4500 0040 3218 4000 4006 0a9e 7f00 0001  E..@2.@.@.......
      0x0010:  7f00 0001 b64c 0d05 8f0d 2e5d 163f caa4  .....L.....].?..
      0x0020:  8018 0200 fe34 0000 0101 080a c24e e2d7  .....4.......N..
      0x0030:  c24e 1217 6865 6c6c 6f20 776f 726c 640a  .N..hello.world.
\end{outputcode}

\begin{enumerate}
\item Tell me the meaning of each option used in the previous command.
  \begin{itemize}
  \item -i:
  \item -nn:
  \item -vvv:
  \item -x:
  \item -X:
  \item -S:
  \item -K:
  \item -s0:
  \end{itemize}

\item Please analyze this captured packet and explain it to me as
  detailed as you can.
  \begin{description}
  \item[Answer:]
  \end{description}
\end{enumerate}

\section{HTTP}

\begin{enumerate}
\item Write a simple script showing how HTTP works (you need
  \texttt{curl});

  \begin{shcode}
    #!/bin/bash

    echo 'Hello, world!'
  \end{shcode}

\item Record your HTTP demo session with \texttt{ttyrec}.
\end{enumerate}

\section{Socket programming}

\subsection{TCP}

\begin{ccode}
/* A simple TCP server written in C */

/* Your code */
\end{ccode}

\begin{ccode}
/* A simple TCP client written in C */

// Your code
\end{ccode}

\subsection{UDP}

\begin{ccode}
/* A simple UDP server written in C */

// Your code
\end{ccode}

\begin{ccode}
/* A simple UDP client written in C */

// Your code
\end{ccode}

\section{Questions}

List at least 5 problems you've met while doing this work. When
listing your problems, you have to tell me:
\begin{enumerate}
\item Description of this problem. For example,
  \begin{itemize}
  \item What were you trying to do before seeing this problem?
  \end{itemize}
\item How did you try solving this problem? For example,
  \begin{itemize}
  \item Did you google? web links?
  \item Did you read the man page?
  \item Did you ask others for hints?
  \end{itemize}
\end{enumerate}

\subsection{Problems}

\begin{enumerate}
\item

\item

\item

\item

\item
\end{enumerate}

\end{document}
%%% Local Variables:
%%% mode: latex
%%% TeX-master: t
%%% End:
